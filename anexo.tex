\appendix
\chapter{Gráficos}

\section{Construcción de gráficos}
\label{c.graf}

La correcta construcción de un gráfico tiene varios puntos: 
\begin{itemize}
  \item ejes de coordenadas con nombre, unidad y flecha que indica dirección
    de valores crecientes;
  \item no se pueden presentar gráficos con líneas verticales continuas;
  \item líneas horizontales continuas solo para funciones; 
  \item no se pueden presentar gráficos en donde no se cumpla la biyección, es
    posible relajar la condición y solicitar que sea inyectiva;
  \item las funciones deben ser continuas y sin saltos, es decir que uno las
    puede graficar sin levantar el lápiz.\footnote{Esta característica depende
    de la situación que será aclarada si no es necesario que se cumpla la
    condición, en particular para la velocidad.}
\end{itemize}
La Figura (\ref{f.mal}) muestra todos estos problemas. Todo lo indicado en
{\color{red}rojo} corresponden a errores en la representación gráfica,
enumerados a continuación:
\begin{itemize}
  \item Eje vertical sin flecha que indique dirección de valores crecientes, sin
    nombre y sin unidad.
  \item Eje horizontal con dos flechas, no queda definida la dirección de
    valores crecientes.
  \item Nombre de eje horizontal sin unidad.
  \item Puntos $a$, $c$, $f$ y $g$ indicados con línea vertical continua. Los
    correctos son los puntos $b$, $d$ y $e$ con línea vertical de trazos.
  \item Puntos $f$ y $g$ indicados con línea continua horizontal y no
    corresponden a una función. La indicación correcta se observa en los puntos
    $a$, $d$ y $e$ con líneas horizontales de trazos.
  \item La función violeta que contiene a los puntos $e$ y $g$ está mal ubicada
    ya que la función deja de ser biyectiva como lo muestran los puntos $d-e$ y
    $f-g$.
  \item La función azul que comienza en el punto $h$ y sería continuación del
    punto $i$ no cumple que se pueda realizar sin levantar la mano. En este
    punto presenta una discontinuidad.
\end{itemize}

\begin{plot}{.9}{f.mal}
  {Representación gráfica con errores comunes.}
  f(x) = 4
  g(x) = 1.2*(x-3)+4
  h(x) = -3*x**2 + 25*x - 43.6
  i(x) = 5*x**2 + 4
  j(x) = -5*(x-5) + 5
  unset border
  unset tics
  #unset key
  set xtics axis mirror
  set ytics axis mirror
  set format xy ""
  # ejes
  set arrow 1 from graph 0, first 0 to graph 1, first 0 heads front lw 3 lc\
  "red" # eje x
  set arrow 2 from 0, graph 0 to 0, graph 1 nohead front lw 3 lc "red"\
  # eje y 
  show arrow 1 ; set label "$t$" at graph .95,.35
  show arrow 2#; set label "$v$ o $x$" at graph 0.2,.98 
  # labels varios
  set label '$a$' at -.7,i(-.5) 
  set label '$b$' at 1.1,f(1.3)+.4
  set label '$c$' at 2.8,f(3)+.4
  set label '$d$' at 3.5,g(3.7)+.4
  set label '$e$' at 3.5,h(3.7)+.4
  set label '$f$' at 4.75,g(4.7)-.2 
  set label '$g$' at 4.8,h(4.7)+.4 
  set label '$h$' at 5.1,j(5)
  set label '$i$' at 5.1,g(5)
  # vertical
  set arrow from -.5,0 to -.5,i(-.5) nohead front lc "red" lt 1 lw 2 
  set arrow from 1.3,0 to 1.3,f(1.3) nohead front lc 0 lt 0 lw 2 
  set arrow from 3,0 to 3,4 nohead front lc "red" lt 1 lw 2
  set arrow from 3.7,0 to 3.7,h(3.7) nohead front  lt 0 lw 2 
  set arrow from 4.7,0 to 4.7,h(4.7) nohead front lc "red" lt 1 lw 2 
  set arrow from 5,5 to 5,g(5) nohead front lc "red" lt 1 lw 3
  set arrow from 5,0 to 5,g(5) nohead front lt 0 lw 2
  # horizontal
  set arrow from -.5,i(-.5) to 0,i(-.5) nohead front lt 0 lw 2
  set arrow from 0,g(3.7) to 3.7,g(3.7) nohead front lt 0 lw 2
  set arrow from 0,h(3.7) to 3.7,h(3.7) nohead front lt 0 lw 2
  set arrow from 0,g(4.7) to 4.7,g(4.7) nohead front lt 1 lc "red" lw 2
  set arrow from 0,h(4.7) to 4.7,h(4.7) nohead front lt 1 lc "red" lw 2
  set arrow from 0,j(5) to 5,j(5) nohead front lt 0 lw 2
  set arrow from 0,g(5) to 5,g(5) nohead front lt 0 lw 2
  # plot
  set xrange [-1.2:7.2]; set yrange [-5:11]
  plot 0 lc 0 t '' \
  , [0:3] f(x) lc 2 t '' \
  , [3:5] g(x) lc 2 t '' \
  , [3.5:5] h(x) lc "violet" t '' \
  , [-1:0] i(x) lc 2 t ''\
  , [5:6.5] j(x) lc "blue" t ''\
  , "<echo '-.5 5.25'" w p pt 7 lc 6 t ''\
  , "<echo '1.3 4'" w p pt 7 lc 6 t '' \
  , "<echo '3 4'" w p pt 7 lc 6 t ''\
  , "<echo '3.7 7.83'" w p pt 7 lc 6 t ''\
  , "<echo '3.7 4.84'" w p pt 7 lc 6 t ''\
  , "<echo '4.7 6.04'" w p pt 7 lc 6 t ''\
  , "<echo '4.7 7.63'" w p pt 7 lc 6 t ''\
  , "<echo '5 5'" w p pt 7 lc 6 t ''\
  , "<echo '5 6.4'" w p pt 7 lc 6 t ''\
\end{plot}



A continuación lo que está correcto en el gráfico:
\begin{itemize}
  \item Toda la función de color verde es correcta.
  \item La indicación de los puntos $b$, $d$ y $e$.
\end{itemize}

Desde el punto de vista de la física, que a un punto del dominio le corresponda
más de un punto en la imagen (por ej. los puntos $d$ y $e$, $h$ e $i$
o líneas verticales continuas) significa que la variable independiente,
generalmente el tiempo, se relaciona con la posibilidad de dos o más
velocidades/posiciones simultáneas, lo que es físicamente incorrecto. {\bf Un
cuerpo no puede tener dos o más velocidades/posiciones distintas en el mismo
instante de tiempo.} Además de esto, un cuerpo no puede pasar de tener una
velocidad/posición a otra distinta sin pasar por todos los valores de 
velocidades/posiciones intermedias, esto sería similar a la
``teletransportación''.\footnote{Para el caso de \mru~ es posible tener gráficos
de $v-t$ donde se presenten estos ``saltos'' en la función.}

Una posible opción correcta para la Figura (\ref{f.mal}) se presenta a
continuación. Notar que el último tramo que comienza en el punto $h$ es
desplazado verticalmente hasta el punto $i$ para que la función pueda ser
realizada sin ``levantar la mano''.

\begin{plot}{.9}{f.bien}
  {Posible representación gráfica de la velocidad o posición en función del
  tiempo sin los errores presentes en la Figura (\ref{f.mal}). Se puede utilizar
  otra unidad según la necesidad.}
  f(x) = 4
  g(x) = 1.2*(x-3)+4
  h(x) = -3*x**2 + 25*x - 43.6
  i(x) = 5*x**2 + 4
  j(x) = -5*(x-5) + g(5)
  unset border
  unset tics
  #unset key
  set xtics axis mirror
  set ytics axis mirror
  set format xy ""
  # ejes
  set arrow 1 from graph 0, first 0 to graph 1, first 0 head front lw 3 lc\
  0 # eje x
  set arrow 2 from 0, graph 0 to 0, graph 1 head front lw 3 lc 0\
  # eje y 
  show arrow 1 ; set label "$t$(h)" at graph .95,.35
  show arrow 2; set label "$v$(km/h)" at graph 0.2,.98 
  set label "o" at graph .25,.92
  set label "$x$(km)" at graph .2,.86
  # labels varios
  set label '$a$' at -.7,i(-.5) 
  set label '$b$' at 1.1,f(1.3)+.4
  set label '$c$' at 2.8,f(3)+.4
  set label '$d$' at 3.5,g(3.7)+.4
  #set label '$e$' at 3.5,h(3.7)+.4
  set label '$f$' at 4.8,g(4.7)-.2 
  #set label '$g$' at 4.8,h(4.7)+.4 
  set label '$i$' at 5.1,g(5)
  # vertical
  set arrow from -.5,0 to -.5,i(-.5) nohead front lc 0 lt 0 lw 2 
  set arrow from 1.3,0 to 1.3,f(1.3) nohead front lc 0 lt 0 lw 2 
  set arrow from 3,0 to 3,4 nohead front lc 0 lt 0 lw 2
  set arrow from 3.7,0 to 3.7,g(3.7) nohead front  lt 0 lw 2 
  set arrow from 4.7,0 to 4.7,g(4.7) nohead front lc 0 lt 0 lw 2 
  set arrow from 5,0 to 5,j(5) nohead front lc 0 lt 0 lw 2
  # horizontal
  set arrow from -.5,i(-.5) to 0,i(-.5) nohead front lt 0 lw 2
  set arrow from 0,g(3.7) to 3.7,g(3.7) nohead front lt 0 lw 2
  #set arrow from 0,h(3.7) to 3.7,h(3.7) nohead front lt 0 lw 2
  set arrow from 0,g(4.7) to 4.7,g(4.7) nohead front lt 0 lc 0 lw 2
  #set arrow from 0,h(4.7) to 4.7,h(4.7) nohead front lt 0 lc 0 lw 2
  set arrow from 0,j(5) to 5,j(5) nohead front lt 0 lw 2
  # plot
  set xrange [-1.2:7.2]; set yrange [-5:11]
  plot 0 lc 0 t '' \
  , [0:3] f(x) lc 2 t '' \
  , [3:5] g(x) lc 2 t '' \
  , [-1:0] i(x) lc 2 t ''\
  , [5:6.5] j(x) lc 2 t ''\
  , "<echo '-.5 5.25'" w p pt 7 lc 6 t ''\
  , "<echo '1.3 4'" w p pt 7 lc 6 t '' \
  , "<echo '3 4'" w p pt 7 lc 6 t ''\
  , "<echo '3.7 4.84'" w p pt 7 lc 6 t ''\
  , "<echo '4.7 6.04'" w p pt 7 lc 6 t ''\
  , "<echo '5 6.4'" w p pt 7 lc 6 t ''\
\end{plot}



%\subsubsection{Gráficos correctos}

%Próximamente...

%El siguiente gráfico se corresponde con la Figura 3-6 \parencite[65]{alvarenga}
%presentada sin errores.

%\begin{plot}{.9}{f.a36}
    {Gráfico sin errores de la Figura 3-6 \parencite[65]{alvarenga}. En color
  están indicados los distintos desplazamientos para cada intervalo.}
  unset border
  unset tics
  #unset key
  set xtics axis mirror
  set ytics axis mirror
  #set format xy ""
  set xtics axis mirror offset 1
  set ytics axis 30,30 mirror 
  # ejes
  set arrow 1 from graph 0, first 0 to graph 1, first 0 head front lw 3 lc\
  0 # eje x
  set arrow 2 from 0, graph 0 to 0, graph 1 head front lw 3 lc 0\
  # eje y 
  show arrow 1 ; set label "$t$(h)" at graph .95,.10
  show arrow 2; set label "$v$(km/h)" at graph 0.13,.98 
  #
  set arrow from 1,0 to 1,90 nohead front lt 0 lw 2
  set arrow from 3,0 to 3,90 nohead front lt 0 lw 2
  set arrow from 4,0 to 4,60 nohead front lt 0 lw 2
  #
  set object 1 rect from 0,0 to 1,30 fc rgb 'blue' fs solid .15 noborder
  set object 2 rect from 1,0 to 3,90 fc rgb 'green' fs solid .15 noborder
  et object 3 rect from 3,0 to 4,60 fc rgb 'purple' fs solid .15 noborder
  set label '$\Delta x_1$' at .4,15 front
  set label '$\Delta x_2$' at 1.9,50 front
  set label '$\Delta x_3$' at 3.4,25 front
  #
  set xrange [-.5:4.5]; set yrange [-5:95]
  plot 0 lc 0 t ''\
  , [0:1] 30 lc 0 t ''\
  , [1:3] 90 lc 0 t ''\
  , [3:4] 60 lc 0 t ''
\end{plot}

%\begin{plot}{.9}{f.a36}
  %  {Gráfico sin errores de la Figura 3-6 \parencite[65]{alvarenga}. En color
  %están indicados los distintos desplazamientos para cada intervalo.}
  %unset border
  %unset tics
  %#unset key
  %set xtics axis mirror
  %set ytics axis mirror
  %#set format xy ""
  %set xtics axis mirror offset 1
  %set ytics axis 30,30 mirror 
  %# ejes
  %set arrow 1 from graph 0, first 0 to graph 1, first 0 head front lw 3 lc\
  %0 # eje x
  %set arrow 2 from 0, graph 0 to 0, graph 1 head front lw 3 lc 0\
  %# eje y 
  %show arrow 1 ; set label "$t$(h)" at graph .95,.10
  %show arrow 2; set label "$v$(km/h)" at graph 0.13,.98 
  %#
  %set arrow from 1,0 to 1,90 nohead front lt 0 lw 2
  %set arrow from 3,0 to 3,90 nohead front lt 0 lw 2
  %set arrow from 4,0 to 4,60 nohead front lt 0 lw 2
  %#
  %set object 1 rect from 0,0 to 1,30 fc rgb 'blue' fs solid .15 noborder
  %set object 2 rect from 1,0 to 3,90 fc rgb 'green' fs solid .15 noborder
  %set object 3 rect from 3,0 to 4,60 fc rgb 'purple' fs solid .15 noborder
  %set label '$\Delta x_1$' at .4,15 front
  %set label '$\Delta x_2$' at 1.9,50 front
  %set label '$\Delta x_3$' at 3.4,25 front
  %#
  %set xrange [-.5:4.5]; set yrange [-5:95]
  %plot 0 lc 0 t ''\
  %, [0:1] 30 lc 0 t ''\
  %, [1:3] 90 lc 0 t ''\
  %, [3:4] 60 lc 0 t ''
%\end{plot}


% -----

\section{Intervalos}
\label{c.interv}

Los intervalos pueden ser \textit{abiertos}, \textit{semi-abiertos o
semi-cerrados} y \textit{cerrados}, y se presentan con distintas expresiones
matemáticas. La Figura (\ref{f.interv}) muestra cada caso y su representación
gráfica.
\begin{plot}{.9}{f.interv}
  {Diagrama correspondiente a los distintos intervalos y su representación
  gráfica. En color se indica el intervalo en cada caso y con
  paréntesis/corchete como se representa.
  }
  unset tics
  unset border
  #set title 'caso (1)' font 'Helvetica,46'
  set label 'intervalo:' at screen -.15, first 1.50 left
  set label '$\cdot$ abierto' at screen -.1,first 1.43 left
  set label '$\cdot$ semi-abierto/cerrado' at screen -.1,first 1.23 left
  set label '$\cdot$ cerrado' at screen -.1,first 1.03 left
  #
  set arrow from -8.5,1.4 to 7,1.4 head lw 2 lc 0
  set label '$x$' at 6.7,1.43
  set label '$-4$' at -4.3,1.43 right
  set label '2' at 2,1.43 left
  set label  "$($" at -4,1.4 center
  set label  "$)$" at 2,1.4 center
  set arrow from -4,1.4 to 2,1.4 nohead lc 2 lw 9
  #
  set arrow from -8.5,1.2 to 7,1.2 head lw 2 lc 0
  set label '$x$' at 6.7,1.23
  set label '$-4$' at -4.3,1.23 right
  set label '2' at 2,1.23 left
  set label  "$($" at -4,1.2 center
  set label  "$]$" at 2,1.2 center
  set arrow from -4,1.2 to 2,1.2 nohead lc 3 lw 9
  #
  set arrow from -8.5,1.0 to 7,1.0 front head lw 2 lc 0
  set label '$x$' at 6.7,1.03
  set label '$-4$' at -4.3,1.03 right
  set label '2' at 2,1.03 left
  set label  "$[$" at -4,1.0 center
  set label  "$]$" at 2,1.0 center
  set arrow from -4,1 to 2,1 nohead lc 1 lw 9
  set label '$a$'  at .5,1.03 center
  set label '$b$'  at -6,1.03 center
  #
  #set arrow 1 from graph 0, first 0 to graph 1, first 0 head lw 2 # eje x
  #set arrow 2 from 0, graph 0 to 0, graph 1 head lw 2	      # eje y 
  #set arrow from -4,0 to 2,0 nohead lc 3 lw 7
  #set arrow from -4,1 to -4,1.4 nohead lc 0 lt 0
  #set arrow from 2,1 to 2,1.4 nohead lc 0 lt 0
  #set arrow from graph 0,first 1.2 to -4,first 1.2 nohead lc 3 lw 7
  #set arrow from 2,first 1 to graph 1,first 1 nohead lc 3 lw 7
  set xrange [-9:7]; set yrange [.9:1.5]
  plot \
  "<echo '-4 1'" w p lc 0 pt 7 ps 1 t '' \
  , "<echo '-4 1.2'" w p lc 0 pt 7 ps 1 t '' \
  , "<echo '-4 1.4'" w p lc 0 pt 7 ps 1 t '' \
  , "<echo '2 1'" w p lc 0 pt 7 ps 1 t '' \
  , "<echo '2 1.2'" w p lc 0 pt 7 ps 1 t '' \
  , "<echo '2 1.4'" w p lc 0 pt 7 ps 1 t '' \
  , "<echo '.5 1'" w p lc 0 pt 1 ps 3 t ''\
  , "<echo '-6 1'" w p lc 0 pt 1 ps 3 t ''\
\end{plot}



A modo de ejemplo se considera el intervalo entre los puntos $x=-4$ y $x=2$.
También se indican los puntos $a$ y $b$ para el último intervalo cerrado (es
equivalente para cualquier intervalo) para ejemplificar la pertenencia o no de
los puntos.

Características de cada intervalo:
\begin{enumerate}
  \item[\bf Abierto:] los intervalos abiertos se indican con paréntesis en ambos
    extremos y con una coma la separación entre ambos valores: $(-4,2)$. También
    se puede utilizar los símbolos mayor/menor ``$< \quad >$'' junto con la
    variable utilizada: $-4 < x < 2$. Ambas expresiones son equivalentes y
    significan que el intervalo está comprendido entre $x=-4$ y $x=2$, pero sus
    puntos extremos no pertenecen al mismo. Esto se expresa del siguiente modo:
    $-4\notin (-4,2)$ y $2\notin (-4,2)$. El significado que el intervalo sea
    abierto hace referencia a que uno se puede acercar al valor 2 tanto como
    quiera pero sin llegar a ese valor.
  \item[\bf Semi-abierto/cerrado:] los intervalos semi-abiertos/cerrados se
    indican con paréntesis en un extremo y corchete en el otro: $(-4,2]$. Para
    nuestro ejemplo es abierto en $x=4$ (no pertenece al intervalo) y cerrado en
    $x=2$. Al ser un intervalo cerrado en un extremo, significa que éste punto
    pertenece: $2 \in (-4,2]$. También se utilizan los símbolos mayor/menor y
    mayor-igual/menor-igual ``$\le \quad \ge$'', cuyo resultado se expresa:
    \mbox{$-4<x\le2$}. El significado que el intervalo sea cerrado significa que
    uno se puede acercar tanto a al valor 2 y tomar éste valor.
  \item[\bf Cerrado:] los intervalos cerrados se indican con corchetes ambos
    extremos: $[-4,2]$ o con los símbolos: $-4\le x \le 2$. Ambos puntos
    pertenecen al intervalo.
\end{enumerate}

Por lo tanto, en base a las características de los intervalos y la pertenencia
de los puntos, se puede analizar $x=a$:
$$
a \in [-4,2],
$$
mientras que para $x=b$:
$$
b \notin [-4,2].
$$

%\clearpage
\section{Función de a tramos, gráficos con GeoGebra}
\label{c.geogebra}

GeoGebra tiene la opción de realizar gráficos de funciones por tramos. Para
realizar esto se puede incorporar cada función por separado o introducir la
función completa. Esto último es lo que se presenta a continuación. Recordar que
GeoGebra es un software matemático y es el usuario quién debe realizar la
interpretación física/matemática del problema. Por este motivo se van a trabajar
con funciones matemáticas que se corresponderían a problemas físicos genéricos.

Se graficará la siguiente función:
\[
f(x) = \left\{
      \begin{array}{ll}
	x+1, & x \in [-2,1] \\
	-3x+6, & x \in (1,3] \\
	(x-5)^2-1, & x \in (3,8] \\
      \end{array}
      \right.
\]

Para realizar la gráfica de la función $f(x)$ se debe utilizar el condicional
{\bf Si} (o \textbf{if} en inglés) que tiene GeoGebra. La sintaxis del comando que se
introduce en la \emph{ENTRADA} de GeoGebra es la siguiente:

\begin{lstlisting}[frame=single,language=octave,caption=\small Código a
utilizar.]
ENTRADA: Si(condición, acción si es verdad, [acción si es falso])

ENTRADA: if(condición, acción si es verdad, [acción si es falso])
\end{lstlisting}
Los dos primeros argumentos son obligatorios, no así el tercero que sería para
el caso en el que la condición es falsa.

Entonces, para el problema se tienen que incorporar tres condicionales ya que se
tiene tres intervalos:

\clearpage
\footnotesize
\begin{lstlisting}[frame=single,language=octave,caption=\small Código que se
ingresa en la \emph{ENTRADA} de GeoGebra para graficar $f(x)$.]
ENTRADA: Si( -2<=x<=1, x+1, Si( 1<x<=3, -3x+6, Si( 3<x<=8, (x-5)^2-1 ) ) )

ENTRADA: if( -2<=x<=1, x+1, if( 1<x<=3, -3x+6, if( 3<x<=8, (x-5)^2-1 ) ) )
\end{lstlisting}
\normalsize
Notar que los intervalos \textbf{no} se escriben con corchetes y paréntesis,
sino con los signos \mbox{mayor/menor} y mayor/menor igual según sea el
intervalo. La gráfica de la función se presenta en la Figura (\ref{f.geogebra}).

\begin{plot}{0.9}{f.geogebra}
  {Gráfico de la función $f(x)$.}
  #unset border
  unset key
  set xtics autofreq 1
  set mxtics 1
  set ytics autofreq 1.0
  set mytics 1
  set grid
  set tics front    #| equivalente a (*) pero en una línea
  #set xtics axis mirror offset 1
  #set ytics axis ("$60$" 0.5,"$-30$" -0.25) mirror offset 0,0.5
  set arrow 1 from graph 0, first 0 to graph 1, first 0 head front lw 3 # eje x
  set arrow 2 from 0, graph .0 to 0, graph 1 head front lw 3		# eje y 
  show arrow 1 #; set label "$t$(h)" at graph .95,.4
  #set xlabel "$t$(h)" offset  graph .46,.53
  show arrow 2#; set label "$v$(km/h)" at graph 0.25,.98 
  #set ylabel "$x$(km)" offset graph .73,.45 rotate by 0
  #
  set yzeroaxis lt -4 lc -4
  set xrange [-3:9]; set yrange [-4:9]
  #
  plot  0 \
  , [-2:1] x+1 lc 0 lw 4 \
  , [1:3] -3*x+6 lc 0 lw 4 \
  , [3:8] (x-5)**2-1 lc 0 lw 4
\end{plot}



Por último, para que aparezca la expresión matemática para $f(x)$ en conjunto
con la gráfica, se introduce lo siguiente:
\begin{lstlisting}[frame=single,language=octave,caption=\small En caso que la 
función se llame $f$ que corresponde al objeto dentro del paréntesis.]
ENTRADA: FórmulaTexto(f) 
\end{lstlisting}
y se observa la siguiente fórmula en el gráfico:
\[
f(x) = \left\{
      \begin{array}{lcl}
	x+1 &:& -2\le x \le 1 \\
	-3x+6 &:& 1<x\le3 \\
	(x-5)^2-1 &:& 3<x\le8 \\
      \end{array}
      \right.
\]



\clearpage

\chapter{Nomenclatura}

Descripción de los símbolos matemáticos utilizados:

\begin{table}[!ht]
  \centering
  \begin{tabular}{cl}
    \hline\hline
    Símbolo & Significado \\
    \hline
    $\in$ & pertenece \\
    $\notin$ & no pertenece$^\S$ \\
    $\exists$ & existencia \\
    $\forall$ & para todo \\
    $\angle$ & ángulo \\
    $(a,b)$ & intervalo abierto desde $a$ hasta $b$ \\
    $(a,b]$ &  intervalo semiabierto, abierto en $a$ y cerrado en $b$ \\
    $[a,b]$ & intervalo cerrado desde $a$ hasta $b$  \\
    $\Rightarrow$ & entonces \\
    \(\equiv \) & equivalente o definición \\
    \(\approx \) o \( \simeq \) & aproximado \\
    \(\therefore\) & por lo tanto \\
    \( x(t) \) & ecuación o función de movimiento para la posición como función
    de tiempo \\
    \( v(t) \) & ecuación o función de movimiento para la velocidad como
    función de tiempo \\
    $a$ & magnitud escalar \\
    $\vec{a}$ & magnitud vectorial \\
    \hline\hline
  \end{tabular}
\end{table}
$^\S$Cuando el símbolo se encuentre tachado, corresponde a su negación.
