\begin{plot}{.9}{f.vss}
  {Representación gráfica de un auto que se mueve con varias velocidades
  constante. Se tiene en cuenta la condición que a las $3\h$ el punto pertenece
  al intervalo de la izquierda y no de la derecha. Se indica esto con punto
  cerrado/punto abierto respectivamente en color rojo. También se indican los
  intervalos sobre un segundo eje de coordenadas temporal.
  }
  unset border
  unset key
  set tics front    #| equivalente a (*) pero en una línea
  set xtics axis mirror offset 1
  set ytics axis ("$60$" 0.5,"$-30$" -0.25) mirror offset 0,0.5
  set arrow 1 from graph 0, first 0 to graph 1, first 0 head front lw 3 # eje x
  set arrow 2 from 0, graph .10 to 0, graph 1 head front lw 3		# eje y 
  show arrow 1 #; set label "$t$(h)" at graph .95,.4
  set xlabel "$t$(h)" offset  graph .46,.4
  show arrow 2#; set label "$v$(km/h)" at graph 0.25,.98 
  set ylabel "$v$(km/h)" offset graph .4,.5 rotate by 0
  set arrow from 3,-.24 to 3,.5 nohead front lc 0 lt 0 lw 4
  set arrow from 5,0 to 5,-.25 nohead front lc 0 lt 0 lw 4
  set arrow from 0,-.25 to 2.98,-.25 nohead front lc 0 lt 0 lw 4
  #
  set arrow from -1.2,-0.5 to 5.7,-0.5 head front lw 3
  set arrow from 0,-.5 to 3,-.5 nohead lc "purple" lw 7 front
  set arrow from 3,-.5 to 5,-.5 nohead lc "red" lw 7 front
  set label '$[$' at 0,-.5 center front 
  set label '$]$' at 3,-.50 center front 
  set label '$($' at 3.02,-.50 center front 
  set label '$]$' at 5,-.50 center front 
  f(x) = .5
  set yzeroaxis lt -4 lc -4
  set xrange [-1.2:5.7]; set yrange [-.5:1]
  #
  plot 0 \
  , [0:3] f(x) lc 0 lw 4 \
  , [3.02:5] -.25 lc 0 lw 4 \
  , "<echo '3 .5'" w p lt 1 ps 1.3 pt 7 lc 7 \
  , "<echo '3 -.25'" w p lt 7 lw 3 ps 1.3 pt 6 lc 7 \
\end{plot}

