    \centering
      %\begin{center}
      \setlength{\unitlength}{1mm}
      \begin{picture}(80,40)
	% --- plano ---
	%\put(-1.5,-1){\line(0,1){5}}
        %\linethickness{0.5mm}
        \put(0,0){\line(3,1){80}}
        %\put(0,0){\circle*{1}}
	%\put(-4,0){$C$}
        \put(0,0){\line(1,0){80}}
        %\put(0,18){\circle*{1}}
        %\put(-4,18){$B$}
	%\put(0,18){\line(3,-2){26.8}}
        %\put(26.8,0){\circle*{1}}
        \qbezier(18,0)(20,3)(18,6)
        %\put(13,1){$30º$}
	\put(13,2){$\alpha$}
        %\put(28,0){$A$}
	% --- cuerpo ---
	\put(30,20){\line(3,1){15}}
	\put(32.9,11.0){\line(-1,3){3}}
	\put(48.0,16.0){\line(-1,3){3}}
	% --- fuerza que tira ---
	\linethickness{0.007mm}
	\put(46.3,21){\line(3,1){30}}
	\linethickness{2mm}
	\thicklines
	\put(46.3,21){\vector(3,1){15}}
	\put(55,27){$\vec{F}$}
	\end{picture}
      %\end{center}
	\label{f.1}
	\caption{Esquema de la situación problemática sobre el plano inclinado.}

