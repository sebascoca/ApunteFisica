  \centering
  \setlength{\unitlength}{1mm}
  \begin{picture}(120,40)
    % --- plano ---
    \put(37,39){(a)}
    \put(0,0){\line(3,1){60}}
    \put(3,3){A}
    \put(0,0){\line(1,0){60}}
    \put(60,-3){B}
    \qbezier(18,0)(20,3)(18,6)
    \put(13,2){$\alpha$}
    % --- direcciones ---
    \put(15,12){\line(1,0){45}}
    \put(60,9){B'}
    \put(36,-3){\line(0,1){35}}
    \put(37,30){C}
    \put(39.9,0.3){\line(-1,3){10}}
    \put(26,30){D}
    % --- ángulos ---
    \qbezier(50,12)(52,14.5)(50,16.7)
    \put(46,13){$\alpha$}
    \qbezier(37.8,7)(42,9)(42,12)
    \put(37.8,8.7){$\beta$}
    \qbezier(36,2)(37,1)(39,2)
    \put(36,3){$\gamma$}
    %
    % --- ubicación en sist. coord. ---
    \put(93,39){(b)}
    \thinlines
    \put(75,23){\vector(1,0){40}}
    \put(115,21){$x$}
    \put(93,0){\vector(0,1){35}}
    \put(90,35){$y$}
    \put(93.5,20.5){o}
    \put(93,23){\circle*{1}}
    \qbezier(87.5,6)(90,4)(93,6)
    \put(89,8){$\alpha$}
    % --- fuerzas ---
    \thicklines
    \put(93,23){\vector(-1,-3){7}}
    \put(85,-3){$\vec{P}$}
    \put(93,23){\vector(0,-1){21.2}}
    \put(95,9){$\vec{P}_y$}
    \put(93,23){\vector(-1,0){7}}
    \put(85,25){$\vec{P}_x$}
    \linethickness{0.07mm}
    \put(93,1.8){\line(-1,0){7}}
    \put(86,23){\line(0,-1){21.2}}
    % --- Sistema de referencia oblicuo ---
    \thinlines
    \put(-10,10){\vector(3,1){20}}
    \put(10,15){$x$}
    \put(0,5){\vector(-1,3){7}}
    \put(-6,25){$y$}
  \end{picture}
  \caption{(a) Diagrama geométrico para identificar el ángulo en el sistema de
  fuerzas. (b) El vector peso, sus componentes y el ángulo utilizado.}
  \label{f.dgeom}

