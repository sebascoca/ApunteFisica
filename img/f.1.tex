 %   \begin{figure}[!ht]
      \centering
      \setlength{\unitlength}{1mm}
      \begin{picture}(110,27)
	% --- eje ---
        \linethickness{0.3mm}
	\put(0,0){\vector(1,0){110}}
	\put(55,-1.5){\line(0,1){3}}
	\put(54,-5){o}
	\put(100,-5){$x(\mathrm{m})$}
	% --- puntos ---
	\linethickness{0.007mm}
	\put(5,15){\line(0,1){2}}
	\put(5,1){\line(0,1){2}}
	\put(4,19){A}
	\put(20,15){\line(0,1){2}}
	\put(19,19){B}
	\put(45,1){\line(0,1){2}}
	\put(44,5){C}
	\put(55,15){\line(0,1){2}}
	\put(54,19){D}
	\put(70,1){\line(0,1){2}}
	\put(69,5){E}
	\put(100,1){\line(0,1){2}}
	\put(99,5){F}
	% --- lineas y distancias ---
	\put(5,16.5){\line(1,0){50}}
	\put(12,13){15}
	\put(35,13){35}
	\put(5,2.5){\line(1,0){95}}
	\put(24,4){40}
	\put(57,4){25}
	\put(84,4){30}
	% --- cotas ---
	%\put(19,15.5){\line(-1,-1){3}}
	%\put(4,2.5){\line(1,1){4}}
	% --- fuerza que tira ---
	%\thicklines
	\end{picture}
	\label{f.1}
	\caption{Esquema que indica distancias entre los puntos. Además se
	presenta un sistema de coordenadas (sistema de referencia).}
  %  \end{figure}

