
% mru vídeo: https://www.educ.ar/recursos/40687/movimiento-rectilineo-uniforme-mru

% secuencia didáctica con modellus+mru: https://www.educ.ar/recursos/14565/movimiento-rectilineo-uniforme-mru

\chapter{Cinemática}

\section{Partícula puntual y movimiento relativo}
\label{c.pmr}

La definición dada en el libro de partícula es: ``Un cuerpo es una
\emph{partícula} cuando sus dimensiones son muy pequeñas en comparación con las
demás dimensiones que participan en el fenómeno'' \parencite[62]{alvarenga}.
%\parencites[c]{DK}[a][v]{DK}
\\

!!!dar la extensión de la definición para cuerpos "grandes"\\

El \textit{concepto} del movimiento relativo planteado en
\citeauthor{alvarenga} (\citeyear[62]{alvarenga}),
deja en claro que la descripción del problema visto por un observador depende
del punto de referencia en el cual se haya situado. Por este motivo es necesario
introducir la noción de \emph{sistema de referencia}.

\subsection{Sistema de referencia}
\label{c.sr}

Un sistema de referencia es un conjunto de convenciones usada por un observador
para poder medir la posición y otras magnitudes físicas de un sistema físico y
de mecánica. Las trayectorias medidas y el valor numérico de muchas magnitudes
son relativas al sistema de referencia que se considere, por esa razón, se dice
que el movimiento es relativo. Sin embargo, aunque los valores numéricos de las
magnitudes pueden diferir de un sistema a otro, siempre están relacionados por
relaciones matemáticas tales que permiten a un observador predecir los valores
obtenidos por otro observador \parencite{w:sr}.

De forma sintética y para el uso práctico, podemos definir a un sistema de
referencia al que se corresponde con un punto físico de observación desde el
cual se describe la posición y movimiento de los cuerpos, generalmente utilizando
un \emph{sistema de coordenadas} espacial (sistema de coordenadas cartesianas -
1D, 2D y 3D) y temporal.

Los sistemas de referencia poseen características:
\begin{itemize}
  \item Independencia respecto al movimiento del cuerpo. Esto implica que la
    descripción del problema físico no depende del sistema de referencia
    (por ejemplo, soltar un piedra descripto por un observador al lado del que
    suelta la piedra, ser el que suelta la piedra o estar arriba de un árbol no
    cambia el proceso física: la piedra cae al suelo).
  \item El tiempo, para todos los sistemas de referencia, es absoluto. Es decir,
    para todos los observadores de un mismo fenómeno, el tiempo en que
    transcurre es el mismo.\footnote{Esto es válido para la mecánica clásica,
    que corresponde a lo estudiado en este curso.}
  \item Las ecuaciones que rigen el movimiento de un cuerpo se cumplen
    equivalentemente, cualquiera sea el sistema de referencia que se observe.
    Es decir, que las ecuaciones utilizadas para describir el movimiento van a
    ser las mismas independiente del sistema de referencia escogido.
\end{itemize}

Además, existen \textit{``libertades''} en la elección del sistema de referencia:
\begin{itemize}
  \item Libertad de elegir el origen temporal. Es decir desde cuando comienzo a
    contar (largo el cronómetro).
  \item Libertad en la elección de donde ubicar el origen del sistema de
    coordenadas, normalmente utilizado para ubicar al observador.
\end{itemize}
Estas libertadas se ``restringen'' cuando en el problema se especifica que se tiene
que describir desde tal lugar y/o contar desde tal otro.
\\


!!!hablar sobre la flecha en los gráficos, su significado y lo mínimo para
comprender el s.c.

\subsubsection{Temporal}
\label{c.srt}

Podemos definir/imaginar un sistema temporal como una recta numérica con un
origen arbitrario, una dirección positiva y la opuesta negativa, además de la
unidad para definir la escala y su unidad (por ej. 1\,h). 

En la vida diaria se identifican muchos de estos sistemas de referencia
temporales, por ejemplo el reloj que puede contar de 0 a 12\,h (o de 0 a 24\,h)
y luego comenzar nuevamente. Podemos deducir que las $23\h$ de ayer corresponden
a las $-1\h$ del día de hoy; y que las $30\h$ de hoy se corresponden con las
$6\h$ de mañana (ambos ejemplos aplicados a relojes de $24\h$). Lo mismo es
aplicable a los meses, los años, etc. Otro ejemplo muy claro y que se
utiliza mucho es la designación AC y DC para indicar fechas antes de Cristo y
después de Cristo, donde el nacimiento sería el origen del sistema.

\subsubsection{Espacial}
\label{c.sre}

Los sistemas de referencia espaciales son básicamente sistemas de coordenadas
cartesianos ortogonales en 1, 2 ó 3 dimensiones, según la necesidad. Dónde
nuevamente será una o varias rectas numéricas perpendiculares entre sí, todas
con un origen que puede o no coincidir (es a elección y necesidad de lo que se
quiera describir) y una dirección positiva indicada por la flecha en el eje,
junto con su partición y las unidades utilizadas (por ej. $1\km$, $5\m$, etc.).

Nuevamente, en la vida diaria estamos constantemente utilizando éste tipo de
sistema de referencia espaciales, por ejemplo las calles y alturas de la ciudad.
Para el caso de Córdoba Capital se corresponde con un sistema de coordenadas
cartesianas ortogonales cuyo origen está ubicado en la esquina de la Plaza San
Martín y el Cabildo. Con el uso de los Sistemas de Posicionamiento Global (GPS),
proporcionado por los teléfonos móviles al día de hoy, se naturaliza otro
sistema utilizado a nivel mundial. Las coordenadas de este sistema poseen
nombres: latitud y longitud, utilizadas para ubicar objetos y trayectorias
en los distintos tipos de planos. Por ejemplo, la Ciudad de Córdoba posee las
siguientes coordenadas geográficas: 
\begin{align*}
  (\phi = 31^{\circ}\, 25'\, 0''\, \mathrm{S}; &
  \: \lambda = 64^{\circ}\, 11'\, 0''\, \mathrm{W}), \\
  \mathrm{o} \\
  (\phi = -31^{\circ}\, 25'\, 0'';&
  \; \lambda = - 64^{\circ}\, 11'\, 0'').
\end{align*}
Donde en el primer conjunto de coordenadas se indica que es latitud Sur y
longitud W (Oeste). Esto se puede obviar al poner el signo menos al comienzo de
las coordenadas como se indica en el segundo par, sabiendo que para el Norte y
Este se considera positivo.


\section{Movimiento Rectilíneo Uniforme (MRU)}
\label{c.mru}

%!!!Se comprende MRU pero el problema surge con la velocidad negativa. Ejemplificar
%y hacer notar que es simplemente una cuestión desde dónde y cuándo contar, dando
%lugar a los sistemas de referencia.\\

Un problema muy común se da, por ejemplo, cuando se quiere viajar a Bs. As. desde
Cba. Se sabe que el viaje toma $7\h$ y que hay que recorrer unos $700\km$. Con
esta información se puede calcular la velocidad del viaje, unos $100\kmh$. 
De manera equivalente, sabemos que si se viaja a la misma velocidad pero durante
$3\h$, se puede calcular que la distancia recorrida, $300\km$. Por último, si
recorrimos $550\km$ a $100\kmh$, se puede saber que se viajó por $5.5\h$, es
decir, por cinco horas y media. Estos son los problemas que se presentan al
trabajar en \mru

La relación matemática que define al \mru~ es \parencite[64]{alvarenga}:
\begin{equation}
  d = v \cdot t.
  \label{e.d.alv}
\end{equation}
De la misma se pueden derivar relaciones para calcular la velocidad $v$ y el
tiempo $t$:
\begin{align}  
  v = & \dfrac{d}{t}; \label{e.v.alv}\\ 
  t = & \dfrac{d}{v}.  \label{e.t.alv}
\end{align}
Con las Ecuaciones (\ref{e.d.alv}), (\ref{e.v.alv}) y (\ref{e.t.alv}) es posible
resolver prácticamente todos los problemas relacionados a \mru~ El problema
surge cuando se introduce el concepto de velocidad negativa. La interpretación
física de la velocidad negativa es que el cuerpo en lugar de ``avanzar'' está
retrocediendo. Es decir, que se mueve en dirección hacia los valores negativos
en un sistema de coordenadas (sistema de referencia). Pero según la Ecuación
(\ref{e.v.alv}) si la velocidad es negativa, entonces $t$ o $d$ deben ser
negativos, uno de ellos, ya que el cociente entre la distancia y el tiempo es la
velocidad, la cual es negativa. Por definición, la distancia $d$ no puede ser
negativa, por lo tanto el tiempo $t$ es negativo$\ldots$ ¿Pero tiene sentido que
el tiempo sea negativo? Según lo visto en (\ref{c.srt}) el tiempo negativo tiene
sentido físico, pero por ejemplo para un problema donde un auto demora $7\h$ en
recorrer $700\km$ jamás dará como resultado una velocidad negativa. Esto se debe
a que al calcular la velocidad, no se especificó ni el momento de salida ni de
llegada, solo que el cuerpo demora cierto tiempo. Podemos decir que en la
Ecuación (\ref{e.v.alv}) no se utiliza el tiempo absoluto, sino el intervalo de
tiempo.

\subsection{Intervalo de tiempo}
\label{c.dt}

Se define intervalo de tiempo $\Delta t$ a la duración de un evento definido
entre dos tiempos específicos, final e inicial:
\begin{equation}
  \Delta t = t_f - t_i,
  \label{e.dt}
\end{equation}
donde $t_f$ se corresponde al tiempo final cuando termina el viaje para nuestro
caso y $t_i$ el tiempo de inicio del viaje. Esta magnitud nunca puede ser
negativa ya que esto implicaría que se está volviendo en el tiempo. Esto está
prohibido por los principios básicos de la mecánica clásica. 

\paragraph{Ejemplo.} 
Para nuestro caso en estudio, el viaje a Bs. As., se puede partir a las $7\h$ y
llegar a las $14\h$, o de manera equivalente salir a las $12\h$ y llegar a las
$19\h$. En ambos casos el intervalo de tiempo fue de $\Delta t = 7\h$.
\finej \\

Por lo tanto, en las Ecuaciones (\ref{e.d.alv}--\ref{e.t.alv}) la variable
tiempo $t$ no tiene sentido, ya que se corresponde con el tiempo absoluto. La
magnitud física que evalúa la duración de un suceso es el intervalo de tiempo
$\Delta t$, y es éste el que se debe utilizar.


\subsection{Desplazamiento o cambio de posición}
\label{c.dx}

\subsubsection{Desplazamiento positivo}

Según lo visto en la sección (\ref{c.dt}) el intervalo de tiempo no es negativo
por definición. Entonces, retomando las Ecuaciones
(\ref{e.d.alv}--\ref{e.t.alv}) jamás podremos obtener una velocidad
negativa$\ldots$ si es que seguimos conservando la distancia $d$.

Si se utiliza $d$ tampoco vamos a poder ubicar a un objeto en un sistema de
coordenadas con valor negativo, ya que la distancia evalúa la longitud del punto
al origen o la longitud entre dos puntos. Para realizar una descripción correcta
la distancia no es útil, en su lugar se puede utilizar la posición del cuerpo
(sus coordenadas espaciales en un sistema de referencia) o el desplazamiento que
realiza el cuerpo. La correcta es el desplazamiento $\Delta x$ y no la posición
de un cuerpo $x$. 

Se define desplazamiento o cambio de posición $\Delta x$ a la diferencia
de posiciones que experimenta un cuerpo:
\begin{equation}
  \Delta x = x_f - x_i,
  \label{e.dx}
\end{equation}
donde $x_f$ se corresponde a la posición que alcanza y $x_i$ a la posición
inicial desde donde partió.

Los siguientes ejemplos demuestran porque se debe usar $\Delta x$ y no $x$.


\paragraph{Ejemplo.} 
Volviendo al problema de viajar a Bs. As. desde Cba., recorriendo $700\km$ en
$\Delta t =7\h$, queremos calcular la velocidad del auto. La misma da
$v=\dfrac{700\km}{7\h}=100\kmh$. Ahora utilicemos un sistema de coordenadas con
origen en Cba. ($x_i=0\km$) y positivo hacia Bs. As. ($x_f=700\km$). ¿Cuál de
las dos coordenadas deberíamos utilizar? ¿Tiene sentido tener la posibilidad de
elegir qué coordenada utilizar? No, lo que tiene sentido es que el
desplazamiento realizado por el auto, que fue de $\Delta x= x_f - x_i =
700\km - 0\km = 700\km$. \finej

\paragraph{Ejemplo.} 
Si ahora se utiliza un sistema de coordenadas con origen en La Cumbre, ubicada a
$100\km$ de Cba. entonces tenemos que Cba. está en $x_i=100\km$ y Bs. As. en
$x_f=800\km$. Si solo se utilizan coordenadas, seríamos incapaces de obtener la
velocidad anterior. Pero el desplazamiento no cambio, sigue siendo de $\Delta x
= x_f - x_i = 800\km - 100\km = 700\km$. Este desplazamiento es el mismo que
utilizando el sistema de coordenadas con origen en Cba.
\finej \\

{\bf Se puede concluir que el desplazamiento que realiza un cuerpo es
independiente del sistema de coordenadas utilizado.}

\subsubsection{Desplazamiento negativo}

Comencemos con un par de preguntas, ¿tiene sentido un desplazamiento negativo?
¿Qué significa? En ningún momento se especificó que el desplazamiento debe ser
positivo, por lo tanto podría ser negativo. Al analizar la Ecuación (\ref{e.dx})
se observa que si el desplazamiento es negativo, significa que la posición
inicial es mayor a la posición final, matemáticamente: $x_i>x_f$. Esto significa
que el cuerpo se está dirigiendo en sentido decreciente en el sistema de
coordenadas, disminuyendo su coordenada.

\paragraph{Ejemplo.} Ahora se considera el caso que el auto vuelve a Cba. desde
Bs. As. El sistema de coordenadas utilizado se corresponde a tomar el origen en
Cba. y contar hacia Bs. As. Entonces, al considerara la ``vuelta'' cambian los
valores de las coordenadas inicial y final. El auto parte de Bs. As.
($x_i=700\km$) y llega a Cba. ($x_f=0\km$). El desplazamiento del auto en este
caso es $\Delta x= 0\km - 700\km=-700\km$. Y por lo establecido en la sección
anterior, al calcular la velocidad se utiliza el desplazamiento en lugar de la
distancia, cuyo resultado es: $v=\dfrac{-700\km}{7\h}=-100\kmh$. ¡La velocidad
es negativa! \finej \\

Por lo tanto, podemos redefinir las Ecuaciones  (\ref{e.d.alv}--\ref{e.t.alv})
con todas las consideraciones y análisis realizados:
\begin{align}  
  \Delta x = &\: v \cdot \Delta t; \label{e.mov} \\
  \Delta t = & \: \dfrac{\Delta x}{v}.  \label{e.t}
\end{align}
Y por último la definición formal que se utilizará de ahora en adelante para la
velocidad:
\begin{equation}
  \boxed{v \equiv \dfrac{\Delta x}{\Delta t} = \dfrac{x_f - x_i}{t_f - t_i}},
  \label{e.v}
\end{equation}
que representa el cambio de posición que realiza un cuerpo en un determinado
intervalo de tiempo.


\subsection{Análisis de unidades}
\label{c.anu}


INCLUIR LO DEL SIMELA, UNIDADES FUNDAMENTALES Y ESPECIFICAR QUE POSICIÓN PUEDE
SER EN º!!!!

https://es.wikipedia.org/wiki/Sistema\_Internacional\_de\_Unidades

Artículo en la wikipedia sobre el sistema internacional de unidades. Es un artículo muy completo, con muy buena información y referencias. Las secciones que les recomiendo leer principalmente son:

    Unidades básicas
    Unidades derivadas
    Normas ortográficas relativas a los símbolos


Al realizar el análisis de unidades de las distintas magnitudes físicas
presentes: tiempo o intervalo de tiempo; distancia, posición o desplazamiento y
velocidad se debe encerrar la variable entre corchetes y solo analizar las
unidades que posea, sin importar el valor numérico: 

\begin{align*}
  [d] = &\: \m,\km,\ldots; \\
  [x] = &\: \m,\km,\ldots; \\
  [\Delta x] = &\: [x_f] - [x_i] = \m, \km, \ldots; \\
  [t] = &\: \s, \h, \text{días},\mathrm{a\tilde{n}o},\ldots; \\
  [\Delta t] =  &\: [t_f] - [t_i] = \s, \h, \text{días},\text{años},\ldots
\end{align*}
Para el caso de la velocidad que es una magnitud que se calcula en base a una
relación, se realiza el análisis de manera equivalente:
\begin{align*}
  [v] = &\: \dfrac{[\Delta x]}{[\Delta t]} = \dfrac{\m,\km,\ldots}{\s,
  \h,\ldots} = \ms, \kmh, \ldots 
\end{align*}
Solo se escribieron las unidades más utilizadas: $\ms$ o $\kmh$. Esto no implica
que se puedan utilizar otras unidades para la velocidad, como por ejemplo:
$\km/\mathrm{a\tilde{n}o}$.

\paragraph{Ejemplo.}
Un cuerpo recorre una distancia $d=60\m$, sus posiciones inicial y final son
respectivamente $x_i=-30\m$ y $x_f=30\m$ y su desplazamiento de $\Delta x=60\m$.
Análisis de unidades:
%\begin{align*}
$$
  [d] =  \m; \qquad
  [x_f] = [x_i] =  \m; \qquad
  [\Delta x] = \m. 
$$
Como todas las variables del problema están en las mismas unidades no hay
problema en trabajar directamente con sus magnitudes.
\finej
%\end{align*}

\paragraph{Ejemplo.} 
Un cuerpo posee una velocidad de $v=60\kmh$ y recorre $180\km$. Se quiere
conocer el tiempo que le toma. Si se equivoca al despejar puede dudar de qué
ecuación utilizar para calcular el intervalo de tiempo:
$$
\Delta t = \dfrac{v}{\Delta x}\qquad \text{o} \qquad \Delta t = \dfrac{\Delta
x}{v}.
$$
Para resolver este problema basta con realizar un análisis de unidades a cada
una de las expresiones. {\bf La relación correcta tiene que dar la misma unidad
de ambos lados de la igualdad}. Se sabe que el resultado debe dar en h, la
unidad de tiempo involucrada en el problema. Primera relación:
\begin{align*}
  [\Delta t] = &\: \left[\dfrac{v}{\Delta x}\right] \\
  h = &\: \dfrac{[v]}{[\Delta x]} \\
  h = &\: \dfrac{\kmh}{\km} = \kmh : \km \\
  h = &\: \dfrac{\cancel{\km}}{\h} \cdot \dfrac{1}{\cancel{\km}} \\
  \textbf{¿}\mathbf{h =} &\: \mathbf{\dfrac{1}{h}}\textbf{?} \qquad
  \longleftarrow \qquad \text{¡No tiene sentido!}
\end{align*}
Por lo tanto la primera relación no es válida. Segunda relación:
\begin{align*}
  [\Delta t] = &\: \left[\dfrac{\Delta x}{v}\right] \\
  h = &\: \dfrac{[\Delta x]}{[v]} \\
  h = &\: \dfrac{\km}{\kmh} = \km : \kmh \\
  h = &\: \dfrac{\cancel{\km}}{1} \cdot \dfrac{h}{\cancel{\km}} \\
  \textbf{¡}\mathbf{h =} &\: \mathbf{h}\textbf{!}
\end{align*}
Por lo tanto ésta última expresión es la correcta para el cálculo del tiempo:
$$
\Delta t = \dfrac{\Delta x}{v} = \dfrac{180\km}{60\kmh} =
\dfrac{180}{60}\dfrac{\km}{\kmh} = 3\h
$$
Respuesta: el auto demoró $3\h$ en recorrer los $180\km$ viajando a una
velocidad de $60\kmh$. \finej \\

{\bf Recordar siempre operar sobre las unidades al igual que con los valores
numéricos.}

%    ejercicios de sr temporal y espacial, mov. rel.

\subsubsection*{Actividades (sistemas de referencia y unidades)}
\small
Resolver los siguientes ejercicios:

% MEJORAR LA REDACCIÓN DE LOS ENUNCIADOS Y VERIFICAR SI TIENEN SENTIDO, POR EJ.
% EL 3.
% DAR EJEMPLOS PREVIOS PARA COMPRENDER QUE DEBEN REALIZAR

\begin{enumerate}
  \item Expresar el horario indicado según el sistema de referencia indicado,
    cuyo sistema de referencia tiene origen a las 12h del día 29 de febrero (12h
    -- 29/02).
    \begin{tasks}(5)
      \task 13h -- 29/02.
      \task 13h -- 28/02.
      \task 13h -- 01/03.
      \task 24h -- 02/03.
      \task 7h -- 29/02.
    \end{tasks}

  \item Expresar en el horario utilizado habitualmente los siguientes eventos
    temporales sabiendo que el sistema de referencia utilizado posee su origen
    23h -- 29/03.
    \begin{tasks}(5)
      \task $-23\h$.
      \task $-50\h$.
      \task 13h.
      \task 24h.
      \task 45h.
    \end{tasks}

  \item Calcular los intervalos de tiempo considerando como instante inicial el
    origen del sistema indicado en el ejercicio 1. ¿Cómo se relacionan estos
    intervalos con los resultados del ejercicio 1.?

  \item Realizar una gráfico (recta) donde ubique todos los puntos del ejercicio
    2. e indique gráficamente los intervalos de tiempo respecto al origen
    utilizado en este ejercicio. ¿Qué conclusión puede obtener?

  \item Dada la información de la siguiente figura, analizar y responder:
    \begin{figure}[!ht]
       %   \begin{figure}[!ht]
      \centering
      \setlength{\unitlength}{1mm}
      \begin{picture}(110,27)
	% --- eje ---
        \linethickness{0.3mm}
	\put(0,0){\vector(1,0){110}}
	\put(55,-1.5){\line(0,1){3}}
	\put(54,-5){o}
	\put(100,-5){$x(\mathrm{m})$}
	% --- puntos ---
	\linethickness{0.007mm}
	\put(5,15){\line(0,1){2}}
	\put(5,1){\line(0,1){2}}
	\put(4,19){A}
	\put(20,15){\line(0,1){2}}
	\put(19,19){B}
	\put(45,1){\line(0,1){2}}
	\put(44,5){C}
	\put(55,15){\line(0,1){2}}
	\put(54,19){D}
	\put(70,1){\line(0,1){2}}
	\put(69,5){E}
	\put(100,1){\line(0,1){2}}
	\put(99,5){F}
	% --- lineas y distancias ---
	\put(5,16.5){\line(1,0){50}}
	\put(12,13){15}
	\put(35,13){35}
	\put(5,2.5){\line(1,0){95}}
	\put(24,4){40}
	\put(57,4){25}
	\put(84,4){30}
	% --- cotas ---
	%\put(19,15.5){\line(-1,-1){3}}
	%\put(4,2.5){\line(1,1){4}}
	% --- fuerza que tira ---
	%\thicklines
	\end{picture}
	\label{f.1}
	\caption{Esquema que indica distancias entre los puntos. Además se
	presenta un sistema de coordenadas (sistema de referencia).}
  %  \end{figure}


    \end{figure}
  
    \begin{enumerate}
      \item Determinar las coordenadas de cada punto en el sistema de referencia
	indicado en la figura.
      \item Calcular el desplazamiento entre cada par consecutivo de puntos
	iniciando de izquierda a derecha y luego de derecha a izquierda. ¿Qué
	relación presentan entre ellos al calcularlos en ambos sentidos?
      \item Tomar como origen de un nuevo sistema de referencia al punto B y
	realizar lo mismo que en el punto anterior. Comparar los resultados de
	ambos ejercicios.
    \end{enumerate}

  \item Incluir MRU con el gráfico planteando un "recorrido" y luego cambiando
  el origen de la descripción!!!

  \item Realizar el análisis de unidades para las siguientes relaciones y
    verificar si son correctas.\footnote{No en el sentido de la física, sino en el
    sentido que las unidades lo sean.} En caso negativo proponer una modificación para
    que sea correcta:
    \begin{tasks}(3)
      \task $d = v^2\cdot t^2$
      \task $v = \dfrac{(\Delta x)^2}{t}$
      \task $t = \dfrac{x}{v}+t_i$
      \task $x_f = v\cdot (t_f - t_i) + x_i$
      \task $x = \dfrac{v}{\Delta t}+v$
      \task $v = \dfrac{\Delta x}{\Delta t} + \Delta t$
    \end{tasks}

  \item Realizar nuevamente los siguientes ejercicios del libro, pero ahora
    indicando el sistema de referencia utilizado, especificar el origen de
    cada uno y utilizar las Ecuaciones (\ref{e.mov}--\ref{e.v}):
    \begin{itemize}
      \item Problemas 6, 7, 10 y 12, recordar realizar el cambio al
	conjunto nuevo de variables ($d \to \Delta x$ y $t \to \Delta t$).
      \item Realizar el gráfico conjunto de los problemas 8 y 9.
      \item Realizar el análisis del ejercicio 11 con las nuevas ecuaciones.
    \end{itemize}

\end{enumerate}
\normalsize

% ---------------

\subsection{Gráficos y Ecuaciones de movimiento}

\subsubsection{Gráfico de la velocidad como función del tiempo}
\label{c.vxt}

Una de las condiciones para que un cuerpo se mueva en \mru, es que su velocidad
se mantengan constante \textit{durante todo su movimiento}. Entonces, al
realizar el gráfico de la velocidad como función del tiempo ($v \times t$, $v$
vs $t$ o $v-t$), este se corresponde matemáticamente al gráfico de una función
constante (Figura \ref{f.fcte}). 

\input{img/f.fcte}

La principal diferencia entre un gráfico matemático y uno físico, es que
normalmente los físicos están acotados a un intervalo temporal y representa un
hecho real. Por ejemplo un auto se desplaza a $60\kmh$ durante $3\h$. La Figura
(\ref{f.v60}) represente la velocidad de este auto y solo está definida para el
intervalo de tiempo comprendido entre $t=0\h$ y $t=3\h$.\footnote{Se evidencia
la elección del origen del sistema de coordenadas temporal. El gráfico podría
haber estado comprendido, por ejemplo, para el dominio temporal entre $t=-1\h$ y
$t=2\h$, pero siempre y cuando se cumpla que $\Delta t = 3\h$.} Observar que el
gráfico se corresponde con las características establecidas en el Apéndice
(\ref{c.graf}) y no presenta errores como en las Figuras 3-5 y 3-6, con líneas
continuas verticales que son incorrectas.
%La representación correcta de la Figura 3-6
%\parencite[65]{alvarenga} se encuentra en el Apéndice (\ref{c.graf}, p.
%\pageref{f.a36}).

\begin{plot}{.9}{f.v60}
  {Representación gráfica de un auto que se mueve a una velocidad constante de
  $60\kmh$ durante $3\h$. Además se indica el desplazamiento realizado por el
  auto, correspondiente al área debajo de la curva.}
  unset border
  unset tics
  set style fill transparent solid 0. noborder
  set object 1 rect from 0,0 to 3,.5 fc rgb 'blue' fs solid .15 noborder 
  set xtics axis mirror offset 1
  set ytics axis ("$60$" 0.5) offset 0,1
  set arrow 1 from graph 0, first 0 to graph 1, first 0 head front lw 3 # eje x
  set arrow 2 from 0, graph 0 to 0, graph 1 head front lw 3		# eje y 
  show arrow 1 #; set label "$t$(h)" at graph .95,.4
  set xlabel "$t$(h)" offset  graph .46,.4
  show arrow 2#; set label "$v$(km/h)" at graph 0.25,.98 
  set ylabel "$v$(km/h)" offset graph .4,.5 rotate by 0
  set arrow from 3,0 to 3,.5 nohead front lc 2 lt 0 lw 4
  set arrow from 0,0 to 0,.5 nohead front lc 2 lt 0 lw 4 #no se ve (?)
  set label '$\Delta x$' at 2,0.7
  set arrow from 2.1,0.66 to 1.5,.3 heads front lc 0 lt 1 lw 2
  f(x) = .5
  set xrange [-1.2:4.2]; set yrange [-.5:1]
  plot 0 lc 0 t '' \
  , [0:3] f(x) lc 2 lw 4 t '' \
  #[0:3] f(x) w filledc y=0  fc "cyan" t '' \
\end{plot}



Del análisis de este gráfico, no se puede decir nada sobre cómo se comportaba el
auto antes de las $0\h$ y después de las $3\h$, no hay información. Solo sabemos
que el auto se desplazó durante $3\h$ a $60\kmh$ y recorrió $180\km$. Este
desplazamiento que realiza el auto se corresponde al área encerrada debajo de la
función, en este caso constante. Los cálculos de la misma, al ser un rectángulo
resultan:
$$
A = \mathrm{base}\times\mathrm{altura} = \Delta t \times v = (t_f - t_i)\times v
= 3h\times 60\kmh = 180\km.
$$

\subsubsection{Ecuación de movimiento para la velocidad como función del tiempo}
\label{c.evxt}

Es posible expresar la información suministrada en la Figura (\ref{f.v60}) a
través de una expresión matemática, una función. La Figura (\ref{f.fcte})
corresponde a la representación gráfica y matemática para una función constante:
$$
f(x)=a,
$$
donde $a$ puede tomar cualquier valor real. La diferencia con física es que la
función $f(x)$ es válida para todo el dominio, mientras que nuestro problema
está acotado a comenzar entre las $0\h$ y $3\h$. Por lo tanto es necesario
suministrar esta información en la ecuación. La forma correcta de realizar esto
es a través del uso de los intervalos para acotar el dominio (ver Apéndice
\ref{c.interv}). La variable que se quiere acotar es la variable independiente,
el tiempo $t$. Además, en física se toma como intervalo cerrado todo el proceso,
ya que se cuenta con información del comienzo y del final, por lo tanto resulta:
$$
t \in [0\h,3\h].
$$
Juntando la información del valor de la velocidad, la ecuación resultante es:
\begin{equation*}
  v(t) = 60\kmh,\quad t \in [0\h,3\h].
\end{equation*}
Esta ecuación se la conoce como la {\bf ecuación de movimiento para la velocidad
como función del tiempo}. La misma indica la velocidad del cuerpo y el intervalo
durante el cuál tuvo esa velocidad.

\paragraph{ Por lo tanto se concluye que la información suministrada por la ecuación de
movimiento para la velocidad como función del tiempo es equivalente a la
presentada en el gráfico de velocidad como función del tiempo.}


\paragraph{Ejemplo.} 
Realizar una descripción de lo observado, dar el desplazamiento total y la
ecuación de movimiento para la velocidad como función del tiempo para el
siguiente gráfico correspondiente a un auto:
\begin{plot}{.9}{f.vss1}
  {Representación gráfica de un auto que se mueve con varias velocidades
  constante.
  }
  unset border
  unset key
  #set grid front   #| son para mandar los números de los
  #unset grid	    #| ejes adelante (*)
  #unset tics
  #set style fill transparent solid 0. noborder
  set tics front    #| equivalente a (*) pero en una línea
  set xtics axis mirror offset 1
  set ytics axis ("$60$" 0.5,"$-30$" -0.25) mirror offset 0,0.5
  #  set ytics axis () offset 0,1
  set arrow 1 from graph 0, first 0 to graph 1, first 0 head front lw 3 # eje x
  set arrow 2 from 0, graph 0 to 0, graph 1 head front lw 3		# eje y 
  show arrow 1 #; set label "$t$(h)" at graph .95,.4
  set xlabel "$t$(h)" offset  graph .46,.4
  show arrow 2#; set label "$v$(km/h)" at graph 0.25,.98 
  set ylabel "$v$(km/h)" offset graph .4,.5 rotate by 0
  set arrow from 3,-.25 to 3,.5 nohead front lc 0 lt 0 lw 4
  set arrow from 5,0 to 5,-.25 nohead front lc 0 lt 0 lw 4
  set arrow from 0,-.25 to 3,-.25 nohead front lc 0 lt 0 lw 4
  set label '$\Delta x_1$' at 1.5,0.3 front
  set label '$\Delta x_2$' at 3.5,-0.18 front
  f(x) = .5
  set yzeroaxis lt -4 lc -4
  set xrange [-1.2:5.7]; set yrange [-.5:1]
  #
  set object 1 rect from 0,0 to 3,.5 fc rgb 'purple' fs solid .15 noborder\
  behind
  set object 2 rect from 3,-.25 to 5,0 fc rgb 'red' fs solid .10 noborder behind
  plot 0 \
  , [0:3] f(x) lc 2 lw 4 \
  , [3:5] -.25 lc 2 lw 4 \
\end{plot}



Interpretación de la información observada: el auto avanza durante $3\h$ a una
velocidad de $v=60\kmh$, luego emprende la vuelta viajando durante $2\h$ y a una
velocidad $v=-30\kmh$. En base a esta información es posible calcular los
desplazamientos parciales en cada intervalo. Se utiliza la Ecuación
(\ref{e.mov}), donde para el primer intervalo se tiene:
$$
t_i=0\h, \quad t_f=3\h, \quad v=60\kmh,
$$
con lo que resulta:
$$
\Delta x_1 = 60\kmh \cdot (3\h-0\h) =  180\km.
$$
Para el segundo intervalo se tiene:
$$
t_i=3\h, \quad t_f=5\h, \quad v=-30\kmh,
$$
entonces:
$$
\Delta x_2 = -30\kmh \cdot (5\h-3\h) =  -60\km.
$$
Por lo tanto, podemos expresar el desplazamiento total como la suma de los
desplazamientos parciales, es decir:
$$
\Delta x = \Delta x_1 + \Delta x_2 = 180\km + (-60\km) = 120\km.
$$

Para la ecuación de movimiento, es necesario conocer la velocidad y el intervalo
de validez de la misma, entonces para cada tramo se tiene:
\begin{align*}
  v(t) &= 60\kmh, \quad t \in [0\h,3\h] \\
  v(t) &= -30\kmh, \quad t \in [3\h,5\h].\\
\end{align*}
Que se reescribir de la siguiente forma como una única ecuación:
\[
v(t) = \left\{ 
\begin{array}{ll}
  60\kmh, & t \in [0\h,3\h] \\
  -30\kmh, & t \in [3\h,5\h] \\
\end{array}
\right.
\]
Con la ecuación de movimiento para la velocidad, se puede saber información de
la velocidad que posee el auto en cierto instante. Por ejemplo, se puede
responder: ¿qué velocidad posee a las 3 horas, 41 minutos y 18 segundos. Lo
primero es ver a qué intervalo pertenece el tiempo deseado y vincular con la
velocidad correspondiente. Ésta es la velocidad que posee el auto en ese
instante. Para nuestra pregunta, vemos que: 
$$
3 \text{ horas, } 41 \text{ minutos y } 18 \text{ segundos} \in [3\h,5\h],
$$
por lo tanto la velocidad en ese instante es $v=-30\kmh$.

Pero todavía hay problemas, ya que no es posible responder con certeza ¿cuál es
la velocidad que posee a las $3\h$? Este instante en particular, según lo
expresado en la ecuación de movimiento, pertenece a ambos
intervalos.\footnote{Ambos intervalos son cerrados sus extremos, por lo tanto,
$t=3\h$ pertenece a ambos intervalos según lo indicado en el Apéndice
(\ref{c.interv}).} Es necesario notar que este problema surge por la suposición
que el auto solo puede tomar valores constantes de velocidad y no variar de
manera continua la velocidad. Entonces al cambiar la velocidad lo realiza de
manera instantánea ``saltando'' de un valor a otro, es decir produciendo una
discontinuidad en la función. Para solucionar este problema se utiliza la
convención que el valor de la función, en nuestro caso velocidad, toma el valor
del que viene en los tiempos previos. Por lo tanto, la velocidad es
$v(3\h)=60\kmh$ y se lee: la velocidad del auto es de $60\kmh$ a las $3\h$.
$v=60\kmh$.
Por último, esta información también tiene que ser presentada en la ecuación de
movimiento y en el gráfico. Según la convención, el punto de cambio de velocidad
pertenece al intervalo de la izquierda, por lo tanto el intervalo de la derecha
es abierto. Con este cambio la ecuación de movimiento para la velocidad como
función de tiempo resulta:
\[
v(t) = \left\{ 
\begin{array}{ll}
  60\kmh, & t \in [0\h,3\h] \\
  -30\kmh, & t \in (3\h,5\h] \\
\end{array}
\right.
\]
Y para el gráfico se indica con punto cerrado la pertenencia y punto abierto que
no pertenece:

\begin{plot}{.9}{f.vss}
  {Representación gráfica de un auto que se mueve con varias velocidades
  constante. Se tiene en cuenta la condición que a las $3\h$ el punto pertenece
  al intervalo de la izquierda y no de la derecha. Se indica esto con punto
  cerrado/punto abierto respectivamente en color rojo. También se indican los
  intervalos sobre un segundo eje de coordenadas temporal.
  }
  unset border
  unset key
  set tics front    #| equivalente a (*) pero en una línea
  set xtics axis mirror offset 1
  set ytics axis ("$60$" 0.5,"$-30$" -0.25) mirror offset 0,0.5
  set arrow 1 from graph 0, first 0 to graph 1, first 0 head front lw 3 # eje x
  set arrow 2 from 0, graph .10 to 0, graph 1 head front lw 3		# eje y 
  show arrow 1 #; set label "$t$(h)" at graph .95,.4
  set xlabel "$t$(h)" offset  graph .46,.4
  show arrow 2#; set label "$v$(km/h)" at graph 0.25,.98 
  set ylabel "$v$(km/h)" offset graph .4,.5 rotate by 0
  set arrow from 3,-.24 to 3,.5 nohead front lc 0 lt 0 lw 4
  set arrow from 5,0 to 5,-.25 nohead front lc 0 lt 0 lw 4
  set arrow from 0,-.25 to 2.98,-.25 nohead front lc 0 lt 0 lw 4
  #
  set arrow from -1.2,-0.5 to 5.7,-0.5 head front lw 3
  set arrow from 0,-.5 to 3,-.5 nohead lc "purple" lw 7 front
  set arrow from 3,-.5 to 5,-.5 nohead lc "red" lw 7 front
  set label '$[$' at 0,-.5 center front 
  set label '$]$' at 3,-.50 center front 
  set label '$($' at 3.02,-.50 center front 
  set label '$]$' at 5,-.50 center front 
  f(x) = .5
  set yzeroaxis lt -4 lc -4
  set xrange [-1.2:5.7]; set yrange [-.5:1]
  #
  plot 0 \
  , [0:3] f(x) lc 0 lw 4 \
  , [3.02:5] -.25 lc 0 lw 4 \
  , "<echo '3 .5'" w p lt 1 ps 1.3 pt 7 lc 7 \
  , "<echo '3 -.25'" w p lt 7 lw 3 ps 1.3 pt 6 lc 7 \
\end{plot}


\finej
\paragraph{Tomar como regla para los intervalos que los mismos comienzan
abiertos a izquierda y son cerrados a derecha, a excepción del primero que es
cerrado a izquierda (los extremos pertenecen por ser el comienzo y fin del
evento físico).\\}

\subsubsection{Gráfico de la posición como función del tiempo}
\label{c.xxt}

Ver próxima sección.

\subsubsection{Ecuación de movimiento para la posición como función del tiempo}
\label{c.exxt}

%!!! construcción, varias formas?

En los siguientes links van a poder acceder a los vídeos de las clases
preparadas para los temas desarrollados en las secciones (\ref{c.xxt}) y
(\ref{c.exxt}):

\begin{itemize}
  \item Parte 1: \href{https://youtu.be/Hz0zQMLYsaM}
  {https://youtu.be/Hz0zQMLYsaM}
  \item Parte 2: \href{https://youtu.be/ADKcXhJsD-w}
  {https://youtu.be/ADKcXhJsD-w} \\
\end{itemize}

%!!!interpretación de v
\paragraph{Ejemplo.} 
Para analizar un caso práctico, vamos a considerar que se conoce la ecuación de
movimiento para la velocidad como función del tiempo:
\[
v(t)= \left\{ 
\begin{array}{ll}
  70\kmh, & t \in [-1\h,2\h] \\
  -80\kmh, & t \in (2\h,3\h] \\
  50\kmh, & t \in (3\h,5\h]\\
\end{array}
\right.
\]
y el dato que se conoce sobre su posición es que el cuerpo se ubica en $50\km$
cuando $t=0.5\h$. Se quiere su ecuación de movimiento para la posición como
función del tiempo y su gráfico correspondiente.\\

Se desea construir su ecuación de movimiento para la posición como función del
tiempo $x(t)$. Al igual que para la velocidad, $x(t)$ debe poseer tres términos
correspondientes a los tres intervalos. Notar que el dato no se corresponde con
el tiempo inicial ni el final del movimiento ni de ninguno de los tiempos en los
cuales cambia la velocidad, sino que es un dato dentro del intervalo de tiempo
en el cual el cuerpo desarrolla su movimiento. De manera genérica es posible
escribir su ecuación de movimiento para la posición:
\[
x(t)= \left\{ 
\begin{array}{ll}
  v_1(t-t_{01}) + x_{01} \\
  v_2(t-t_{02}) + x_{02} \\
  v_3(t-t_{03}) + x_{03} \\
\end{array}
\right.
\]
donde los subíndices 1, 2 ó 3 hacen referencia a cada intervalo. De la ecuación
de la velocidad se tienen datos de las mismas y cómo el dato se corresponde
con un dato en el primer intervalo, es posible reescribir la ecuación anterior:
\[
x(t)= \left\{
\begin{array}{ll}
  70\kmh(t-0.5\h)+50\km, & t \in [-1\h,2\h] \\
  -80\kmh(t-t_{02})+x_{02}, & t \in (2\h,3\h] \\
  50\kmh(t-t_{03})+x_{03}, & t \in (3\h,5\h]\\
\end{array}
\right.
\]
Por lo tanto, el problema se reduce a encontrar los pares de puntos $(t_0,x_0)$
correspondientes a cada intervalo. 

Para el caso de los instantes de tiempo, se recomienda tomar el instante inicial
de cada intervalo, que además se corresponde con los cambios de velocidad que
posee el cuerpo. Esto da como resultado:
\[
t_{02} = 2\h \qquad \textrm{ y } \qquad t_{03}=3\h.
\]

Para obtener las posiciones correspondientes a los instantes recién obtenidos,
es importante recordar que \textbf{la posición de un cuerpo siempre es continua}.
Esto quiere decir que en los instantes en los cuales cambia de velocidad, la
posición a la que llega el cuerpo con una velocidad es la misma posición de la
que sale pero con otra velocidad. Físicamente, se plantea las siguientes
ecuaciones, una por cada tiempo en el cual se da un cambio de la velocidad (dos
en el problema):
\[
x_1(2\h) = x_2(2\h) \qquad \textrm{ y } \qquad
x_2(3\h) = x_3(3\h),
\]
donde nuevamente el subíndice hace referencia a la ecuación en su
correspondiente intervalo. Por lo tanto para el primer y segundo intervalo, que
es para $t=2\h$, se tiene:
\[
70\kmh(\mathbf{2\h}-0.5\h)+50\km = -80\kmh(\mathbf{2\h}-2\h)+x_{02},
\]
Donde en negrita se remarcó la evaluación de la función de posición en el
instante en que cambia la velocidad. De esta ecuación es posible despejar la
posición $x_{02}$ correspondiente a $t=2\h$:
\[
x_{02} = 155\km.
\]
De manera equivalente se plantea para la intersección del segundo y tercer
intervalo:
\[
-80\kmh(\mathbf{3\h}-2\h)+155\km = 50\kmh(\mathbf{3\h}-3\h)+x_{03},
\]
cuyo resultado da:
\[
x_{03} = 75\km.
\]
Con todos los datos obtenidos, es posible escribir la ecuación para la posición
como función del tiempo y su gráfico:
\[
x(t)= \left\{
\begin{array}{ll}
  70\kmh(t-0.5\h)+50\km, & t \in [-1\h,2\h] \\
  -80\kmh(t-2\h)+155\km, & t \in (2\h,3\h] \\
  50\kmh(t-3\h)+75\km, & t \in (3\h,5\h]\\
\end{array}
\right.
\]
\begin{plot}{0.9}{f.ejemploep}
  {Ecuación de movimiento del ejemplo. Se indica el dato dado en el problema.}
  #unset border
  unset key
  set xtics autofreq 2
  set mxtics 2
  set ytics autofreq 30.0
  set mytics 3
  set grid
  set tics front    #| equivalente a (*) pero en una línea
  #set xtics axis mirror offset 1
  #set ytics axis ("$60$" 0.5,"$-30$" -0.25) mirror offset 0,0.5
  set arrow 1 from graph 0, first 0 to graph 1, first 0 head front lw 3 # eje x
  set arrow 2 from 0, graph .0 to 0, graph 1 head front lw 3		# eje y 
  show arrow 1 #; set label "$t$(h)" at graph .95,.4
  set xlabel "$t$(h)" offset  graph .46,.39
  show arrow 2#; set label "$v$(km/h)" at graph 0.25,.98 
  set ylabel "$x$(km)" offset graph .33,.45 rotate by 0
  set label "$(0.5h;50km)$" at 0.6,40
  set arrow from 0.5,-60 to 0.5,50 nohead front lt 0 lw 2
  set arrow from -2,50 to 0.5,50 nohead front lt 0 lw 2
  #
  set yzeroaxis lt -4 lc -4
  set xrange [-2:6]; set yrange [-60:180]
  #
  plot  0 \
  , [-1:2] 70*(x-.5) + 50 lc 0 lw 4 \
  , [2:3] -80*(x-2) + 155 lc 0 lw 4 \
  , [3:5] 50*(x-3) + 75 lc 0 lw 4 \
  , "<echo '0.5 50'" w p ps 2 pt 7 lc 6 t '' \
\end{plot}


Observar que la posición como función del tiempo se corresponde con el gráfico
de una función que se puede realizar sin levantar la mano, es decir que es
una función continua.
\finej


\subsubsection*{Actividades (gráficos y ecuaciones de movimiento)}
\small
Resolver los siguientes ejercicios:

\begin{enumerate}
  \item Realizar el gráfico de $v-t$ correcto y luego escribir la ecuación de
    movimiento de la velocidad para las Figuras 3-5 y 3-6
    \parencite[65]{alvarenga}.
  \item Para la Figura 3-8 y la Figura del Ejercicio 12
    \parencite[68]{alvarenga} realizar el cálculo de la velocidad, construir el
    gráfico y escribir la ecuación de movimiento para la velocidad como función
    del tiempo.
  \item Realizar el gráfico para la siguiente ecuación de movimiento de la
    velocidad como función del tiempo:
    \[ v(t) = \left\{
      \begin{array}{ll}
	-30\kmh, & t \in [-2\h,1\h] \\
	80\kmh, & t \in (1\h,4\h] \\
	50\kmh, & t \in (4\h,6.5\h] \\
	-20\kmh, & t \in (6.5\h,6.75\h] \\
	70\kmh, & t \in (6.75\h,9\h] \\
      \end{array}
      \right.
    \]
    Calcular el desplazamiento total que realiza.
    \item Construir el gráfico y la ecuación de movimiento de la posición como
    función del tiempo correspondiente al punto 1. considerando como dato para
    cada uno:
    \begin{enumerate}
      \item la posición del cuerpo es $0\km$ cuando el tiempo es $0\h$;
      \item posición de $15\km$ en $0.5\h$.
    \end{enumerate}
  \item Escribir la ecuación de movimiento $x(t)$ para los puntos del ejercicio
    2.
  \item Construir la ecuación de movimiento $x(t)$ y el gráfico correspondiente
    para el problema de la actividad 3., considerar los siguientes datos:
    \begin{enumerate}
      \item $(t_0;x_0)=(-2\h;-40\km)$;
      \item $(t_0;x_0)=(0\h;40\km)$;
      \item $(t_0;x_0)=(0\h;-40\km)$;
    \end{enumerate}
  \item Tomar $x(t)$ construida para el caso c) y responder:
    \begin{enumerate}
      \item Determinar la posición del cuerpo para los siguientes instantes:
	$t=0\h, \quad t=-1.25\h$, $t=5.5\h \textrm{ y } t=8,\!\overline{3}\h$.
	% casi como debería ser: 5,\!\stackrel\frown{5}$.
      \item Determinar en qué instante de tiempo el cuerpo se encuentra en
	$x=20\km$, $x=-70\km$ y en $x=450\km$. ¿Qué conclusiones puede obtener
	de éste ejercicio?
      \item Calcular el desplazamiento para el intervalo: $[2\h,7\h]$.
      \item Determinar la velocidad que el cuerpo desarrolla durante el
	intervalo especificado en el punto anterior.
    \end{enumerate}

\end{enumerate}
\normalsize

En el Anexo (\ref{c.geogebra}) van a encontrar la metodología para realizar
gráficos de funciones de a tramos en GeoGebra y puedan verificar si los gráficos
que realizaron están correctos.

\subsection{Velocidad media, rapidez y velocidad instantánea}
\label{c.vels}

En los siguientes links van a poder acceder a los vídeos de las clases
preparadas para los temas desarrollados en la sección (\ref{c.vels}):

\begin{itemize}
  \item Parte 1: Velocidad media, rapidez y velocidad instantánea (método
    gráfico)\\
    \href{https://youtu.be/z2VE5GNGZqQ}{https://youtu.be/z2VE5GNGZqQ}
  \item Parte 2: Velocidad instantánea (método analítico)\\
    \href{https://youtu.be/ae\_HijqGzMA}{https://youtu.be/ae\_HijqGzMA} \\
\end{itemize}


MENCIONAR QUE LA VELOCIDAD INSTANTÁNEA ES LA VELOCIDAD MEDIA CUANDO Dt=0, ESTO
ES LO QUE SE CALCULA.

\paragraph{Ejemplo práctico.}
Una forma práctica de entender la diferencia entre la velocidad instantánea y la
media y el proceso por el cual es posible calcular la velocidad instantánea a
través de la velocidad media es al utilizar el poder de cómputo de las
computadoras.

Suponga que se quiere calcular la velocidad instantánea de un cuerpo con la
siguiente ecuación de movimiento:
\[
x(t) = 5t^3 - 2t^2 +7t-19,
\]
en el instante $t=2.5\h$ ($[x]=\km$).

Se construye una tabla y con la ayuda de una planilla de cálculo, se computa el
valor de la velocidad media para distintos valores del intervalos de tiempos,
cada vez más chicos. Cuando el intervalo de tiempo tienda a cero, $\Delta t\to
0$, se obtiene la velocidad instantánea.

En la siguiente tabla se presentan algunos casos:
\begin{table}[!h]
\centering
\footnotesize
\hspace*{-0.7cm}
\begin{tabular}{ccc|cccc|l}
\hline\hline
 &&& 1º paso & 2º paso & 3º paso & 4º paso \\
$t_i$ & $\Delta t$ & $t_f$ & $x_i$ & $x_f$ & $\Delta x$ & 
$\overline{v}=\sfrac{\Delta x}{\Delta t}$ & 
$\left|v(2.5\h) - \bar{v}\right|$ \\
\hline
$2.5$ & 1 & $3.5$ & $64.125$ & $195.375$ & $131.25$ & $131.25$ & $40.5$ \\
$2.5$ & $0.5$ & 3 & $64.125$ & 119 & $54.875$ & $109.75$ & 19 \\
$2.5$ & $0.125$ & $2.625$ & $64.125$ & $76.03320313$ & $11.90820313$ &
$95.265625$ & $4.515625$ \\
$2.5$ & $0.003960625$ & $2.50390625$ & $64.125$ & $64.48003417$ & 
$0.3550341725$ & $90.88874817$ & $0.1387481689$ \\
$2.5$ & $0.000244140625$ & $2.500244141$ & $64.125$ & $64.14715788$ & 
$0.02215787776$ & $90.75866729$ & $0.008667290211$ \\
\hline
\end{tabular}
\normalsize
\caption{Ejemplo práctico del cómputo de la velocidad instantánea a través de la
velocidad media; $[t]=\h$ y $[x]=\km$. Se indican los primeros 4 pasos para el
cómputo de la velocidad instantánea y la diferencia con la velocidad
instantánea ($v(2.5\h)=90.75\kmh$).}
\label{t.velinst}
\end{table}

Se observa en la Tabla (\ref{t.velinst}) como la velocidad media se aproxima a
la velocidad instantánea a medida que el intervalo de tiempo se reduce, esto se
evidencia en el valor absoluto entre las diferencias de velocidades expuesto en
la última columna. Se especifican los cuatro primeros pasos del cálculo de la
velocidad instantánea para cada intervalo de tiempo. La posición inicial $x_i$
no cambia su valor debido a que el instante inicial $t_i$ es siempre el mismo.

La Figura (\ref{f.velinst}) presenta los resultados de la Tabla
(\ref{t.velinst}) de manera gráfica. En negro corresponde a la función de
posición $x(t)$, en verde, violeta y azul se presentan las rectas
correspondientes a cuerpos con velocidades medias de $131.25\kmh$, $109.75\kmh$
y $95.27\kmh$ respectivamente. Por último, la recta tangente en el punto es 
roja.


\begin{plot}{0.9}{f.velinst}
  {Función posición $x(t)$ en el intervalo de interés. Se presentan las
  distintas velocidades medias calculadas en la Tabla (\ref{t.velinst}) en color
  verde, violeta y azul, junto con la velocidad instantánea en color rojo. En
  el recuadro inferior presenta en detalla la función de posición junto con la
  recta tangente.}
  #unset border
  unset key
  set xtics autofreq .5
  set mxtics 2
  set ytics autofreq 30.0
  set mytics 3
  set grid
  set tics front    #| equivalente a (*) pero en una línea
  set xlabel "$t$(h)" #offset  graph .46,.39
  set ylabel "$x$(km)" # offset graph .33,.45 rotate by 0
  #
  set multiplot
  #set size 1,1
  #set origin 0,0
  set yzeroaxis lt -4 lc -4
  set xrange [2.25:3.5]; set yrange [40:200]
  #
  f(x)=5*x**3 - 2*x**2 + 7*x - 19
  x0=64.125
  t0=2.5
  plot  0 \
  , [2.3:2.4] 190 w l lc 2 lw 2  \
  , [2.3:2.4] 175 w l lc 9 lw 2  \
  , [2.3:2.4] 160 w l lc 6 lw 2  \
  , [2.3:2.4] 135 w l lc 7 lw 2  \
  , 131.25*(x-t0)+x0 w l lc 2 \
  , 109.75*(x-t0)+x0 w l lc 9 \
  , 95.265625*(x-t0)+x0 w l lc 6\
  , f(x) w l lc 0 lw 3\
  , 90.75*(x-t0)+x0 w l lc 7\
  # subplot
  set size 0.35,0.35
  set origin 0.60,0.14
  set xrange [2.2:2.8]; set yrange [40:89]
  set label '$\bar{v}=131.25\,$km/h' at .95,193 font ",5"
  set label '$\bar{v}=109.75\,$km/h' at .95,175 font ",5"
  set label '$\bar{v}=95.27\,$km/h' at .95,159 font ",5"
  set label '$v(2.5\text{h})=90.75\text{km/h}$' at .95, 133 font ",5"
  #
  unset ylabel
  unset xlabel
  plot  0 \
  , f(x) w l lc 0 lw 3\
  , 90.75*(x-t0)+x0 w l lc 7
  unset multiplot
\end{plot}



En el siguiente 
\href{https://drive.google.com/open?id=1bQr\_anDE0co2UJfpLdKI5xJT9iGHT3MXe3RD\_4yxqZQ}
{archivo} acceden a una planilla de cálculo para verificar lo establecido
en la Tabla (\ref{t.velinst}) y probar con otros valores de tiempo para calcular
la velocidad instantánea (otro valor del instante inicial $t_i$), distintos
intervalos de tiempos y formas de disminuir el mismo.

\subsubsection*{Actividades (velocidades)}
\small
Resolver los siguientes ejercicios:
\begin{enumerate}
  \item Para los siguientes gráficos $x(t)$ calcular la velocidad según se
    indique. En el caso de velocidades medias representarlas en el gráfico y
    explicar que representan físicamente; para rapidez explicar que representa:
    \begin{enumerate}
      \item Velocidad media en $t\in [-6\h,6\h]$ y $t\in[-2\h,5\h]$; y velocidad
	instantánea en: $-6\h$, $0\h$, $3\h$ y $5\h$.
	\input{img/f.vi1}
      \item Rapidez en $t\in[-6\h,2\h]$ y $t\in[4\h,8\h]$; y velocidad
	instantánea en: $-4\h$, $0\h$ y $4\h$.
	\begin{plot}{1.1}{f.vi2}
  {Actividad 1.\textit{b}}
  #unset border
  unset key
  set xtics autofreq 2
  set mxtics 2
  set ytics autofreq 3.0
  set mytics 3
  set grid
  set tics front    #| equivalente a (*) pero en una línea
  #set xtics axis mirror offset 1
  #set ytics axis ("$60$" 0.5,"$-30$" -0.25) mirror offset 0,0.5
  set arrow 1 from graph 0, first 0 to graph 1, first 0 head front lw 3 # eje x
  set arrow 2 from 0, graph .0 to 0, graph 1 head front lw 3		# eje y 
  show arrow 1 #; set label "$t$(h)" at graph .95,.4
  set xlabel "$t$(h)" offset  graph .46,.53
  show arrow 2#; set label "$v$(km/h)" at graph 0.25,.98 
  set ylabel "$x$(km)" offset graph .73,.45 rotate by 0
  #
  f(x) = -.025*x**3 - .23*x**2 + 2*x + 5
  set yzeroaxis lt -4 lc -4
  set xrange [-10:10]; set yrange [-15:15]
  #
  plot  0 \
  , f(x) lc 0 lw 4 \
\end{plot}


      \item Velocidad media en $t\in[-4\h,5\h]$; rapidez en $t\in[-4\h,5\h]$; y
	velocidad instantánea en: $-6\h$, $-1\h$, $5\h$ y $7\h$.
	\input{img/f.vi3}
    \end{enumerate}
  \item Calcular la velocidad instantánea en $-2\h$, $-1\h$, $0\h$ y $3\h$ para
    las siguientes ecuaciones de movimiento (donde $[x]=km$ y $[t]=h$):
    \begin{enumerate}
      \item $x(t)=30t+15$.
      \item $x(t)=30t^2+15$.
      \item $x(t)=-5t^2+10t-5$.
    \end{enumerate}
  \item Para las ecuaciones de movimiento del ítem anterior calcular la
    velocidad media para los siguientes intervalos de tiempo: $[-2\h,-1\h]$,
    $[-2\h,3\h]$ y $[-2\h,2\h]$. ¿Cuál de las tres velocidades medias se
    aproxima más al valor de la velocidad instantánea en $t=-2\h$? Justificar.
  \item Un auto se desplaza durante $3\h$ a $75\kmh$, luego se deteniene y al
  cabo de media hora sigue su viaje. Cómo va demorado y tiene que llegar en
  $2,5\h$ a su destino, decide ir más rápido para poder recorrer los $220\km$
  restantes.
  Responder:
  \begin{enumerate}
    \item Construir el gráfico $v-t$.
    \item Escribir la ecuación de movimiento para $v(t)$. 
    \item Realizar el gráfico $x-t$ al suponer que parte a las $7:30h$.
    \item A continuación construir la ecuación de movimiento $x(t)$. 
    \item Calcular la velocidad media de todo el viaje. ¿Cómo interpreta este
    resultado?
    \item Realice nuevamente la resolución del problema suponiendo que la
    ubicación de la cual parte es $x_0=153\km$ en un sistema de referencia que
    cuenta en sentido contrario a donde se dirige.
    \end{enumerate}
  \item Calcular la velocidad instantánea en $-1\s$, $-2\s$ y $1\s$ para las
  siguientes ecuaciones de movimiento (donde $[x]=m$ y $[t]=s$):
    \begin{enumerate}
      \item $x(t)=3t^3-5t^2$
      \item $x(t)=\dfrac{2}{3}t^3 -\dfrac{3}{2}t^2 +10t$
      \item $x(t)=\dfrac{2}{t+3}$
      \item $x(t)=\dfrac{5}{t^2} + 4t$
    \end{enumerate}
\end{enumerate}
\normalsize


% ---------------

\section{Movimiento Rectilíneo Uniformemente Variado (MRUV)}

% caida libre en vácio
%https://www.youtube.com/watch?v=E43-CfukEgs

\subsection{Caída Libre y Tiro Vertical}
\label{c.cltv}

En el siguiente link van a poder acceder al vídeo de la clase
preparadas para el tema desarrollado en la sección (\ref{c.cltv}). Primero se
recomienda estudiar el capítulo 3.5 (\citeauthor{alvarenga}). Tengan presente
que el tratamiento del Tiro Vertical no se presenta de manera detalla en el
libro:

\begin{itemize}
  \item Aplicación de MRUV: Caída Libre y Tiro Vertical. \\
    \href{https://youtu.be/8vFQ6gut3Do}{https://youtu.be/8vFQ6gut3Do}
\end{itemize}


% ---------------

\section{Movimientos en más de una dimensión}
\label{c.2d}

En el estudio del movimiento de los cuerpos, el paso siguiente es considerar que
el movimiento es en más de una dimensión. Por ejemplo puede ser cuando se arroja
una piedra o una jabalina, dar un pase en cualquier deporte, etc. En todos los
casos es nuestro cerebro el encargado de resolver el problema y conseguir el
objetivo deseado, hacer llegar el cuerpo a un punto específico, sin siquiera
darnos cuenta. Para esto es necesario estimar la dirección, sentido y la
magnitud de la velocidad, fuerza o aceleración que se le otorgue al cuerpo. Por
lo tanto, ahora es necesario dar más información de las magnitudes físicas y
representarlas por vectores. Por tal motivo primero se estudia la adición de
vectores.

AGREGAR EL TRATAMIENTO COMO COORDENADAS DE LOS VECTORES (AVERIGUAR NOMBRE)!!!

\subsection{Adición de vectores}
\label{c.sv}

En el siguiente link van a poder acceder al vídeo de la clase
preparadas para el tema desarrollado en la sección (\ref{c.sv}). Primero se
recomienda estudiar los capítulos 4.1 y 4.2 (\citeauthor{alvarenga}):

METODO DEL PARALELOGRAMO Y DE LA POLIGONAL!!!

\begin{itemize}
  \item Adición de vectores que no son ni paralelos ni perpendiculares entre sí
  (métdos gráfico y analítico)\\
    \href{https://youtu.be/FMqEXO3b8E4}{https://youtu.be/FMqEXO3b8E4}
\end{itemize}

%INCLUIR EJEMPLOS GRÁFICOS Y ANALÍTICOS!!!  <- hay en el vídeo


\subsubsection*{Actividades (vectores)}
\small
\begin{enumerate}
  \item Dados los siguientes vectores:\footnote{El ángulo especifica la dirección
  y sentido del vector.} $\vec{A}=5\m$ con $\alpha=0\grm$,
  $\vec{B}=13\m$ con $\beta=90\grm$, $\vec{C}=3\m$ con $\gamma=180\grm$,
  $\vec{D}=5\m$ con $\delta=60\grm$ y $\vec{E}=7\m$ con $\epsilon=200\grm$,
  realizar las siguientes adiciones con el método gráfico:
    \begin{enumerate}
    \item $\vec{A}+\vec{C}$
    \item $\vec{D}+\vec{B}$
    \item $\vec{A}+\vec{C}+\vec{E}$
    \item $\vec{A}+\vec{C}+\vec{E}+\vec{B}+\vec{D}$
    \item $3\vec{D}$
    \item $\vec{A}+2\vec{C}$
    \item $\vec{A}-\vec{C}$
    \end{enumerate}
  \item A continuación realizar la adición analítica con los mismos vectores del
  ejercicio anterior:
    \begin{enumerate}
    \item $\vec{C}+\vec{A}$
    \item $\vec{D}+\vec{C}$
    \item $\vec{D}+\vec{C}+\vec{B}$
    \item $\vec{B}+\vec{E}+\vec{D}$
    \item $\vec{C}-\frac{1}{2}\vec{A}$
    \end{enumerate}
\end{enumerate}
\normalsize

\subsection{Composición de movimientos}
\label{c.cm}

A continuación está el link para acceder al vídeo de la clase
preparadas para el tema desarrollado en la sección (\ref{c.cm}):

\begin{itemize}
  \item Composición de movimientos, aplicación a dos casos prácticos \\
    \href{https://youtu.be/R1N\_fhKb4pY}{https://youtu.be/R1N\_fhKb4pY}
\end{itemize}

\subsubsection*{Actividades (composición de movimientos)}
\small
\begin{enumerate}
  \item Calcular el tiempo de vuelo y el alcance que realiza un cuerpo que es
  arrojado con $\vec{v}=72\kmh$, con una inclinación {\bf respecto de la
  vertical} de:
    \begin{enumerate}
      \item $\alpha = 30\grm$
      \item $\alpha = 40\grm$
      \item $\alpha = 50\grm$
      \item $\alpha = 60\grm$
      \item $\alpha = 45\grm$
    \end{enumerate}
  \item Realizar un análisis de los datos obtenidos en el punto anterior y
  determinar bajo qué condiciones se alcanza (justificar sus respuestas):
    \begin{enumerate}
      \item máxima distancia horizontal;
      \item máxima altura vertical;
      \item máximo tiempo de vuelo;
      \item máximo tiempo de vuelo y altura vertical;
      \item máximo tiempo de vuelo y distancia horizontal;
      \item ¿Será posible cumplir que el cuerpo posea simultaneamente máxima
      altura vertical y distancia horizontal?
      \item Si se aplica este análisis a los deportes, por ejemplo lanzamiento
      de bala o jabalina, ¿cuáles son las condiciones óptimas que se tienen que
      cumplir para ganar la competencia?
    \end{enumerate}
  \item En el lanzamiento de disco un competidor obtiene los siguientes datos de
  su lanzamiento:
    \begin{itemize}
      \item Alcance $76.80\m$ (record mundial femenino).
      \item Tiempo de vuelo $3.78\s$.
    \end{itemize}
    Responder:
    \begin{enumerate}
    \item Determinar el vector velocidad que el deportista le otorga al disco.
    \item ¿Puede alcanzar los mismos resultados con otro vector velocidad?
    Justificar.
    \item Si el tiempo de vuelo no es un dato exacto, sino que es un intervalo
    de tiempo centrado en el valor expuesto, ¿qué sucede en este caso con el
    vector velocidad? Realizar el análisis.
    \end{enumerate}
\end{enumerate}
\normalsize


