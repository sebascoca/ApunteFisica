
\chapter{Dinámica}

En esta capítulo estudiaremos las causas que afectan y/o modifican los
movimientos de un cuerpo como la influencia externa de todos los otros cuerpos o
el sistema donde se encuentra.
Estas interacciones externas son lo que se conocen como \emph{fuerzas} y dan
origen a distintos tipos de movimientos.

\section{Leyes de Newton}
\label{c.ln}

\subsection{1º Ley de Newton o Ley de Inercia}
\label{c.1ln}

\begin{tcolorbox}[colback=black!5!white,colframe=red!35!black]
La 1º Ley de Newton establece que un cuerpo mantiene su estado de movimiento a
menos que experimente una interacción externa.
\end{tcolorbox}

Con estado de movimiento se refiere a que mantiene la velocidad que éste posee,
ya sea que esté en reposo ($\vec{v}=0$) o en \mru~ ($|\vec{v}|=$cte.). Cuando
cambia su estado de movimiento, ya sea por cambio en la magnitud de la velocidad
($|\vec{v}|\neq$cte.) o en su dirección es por una fuerza que produce la
interacción.

También se conoce como Ley de Inercia, donde la inercia es la capacidad de un
cuerpo de mantener (de no querer cambiar) su estado de movimiento, es decir de oponerse a
cambiar su estado de movimiento.

Dentro del estudio de las Leyes de Newton, es muy importante realizar el
análisis previo del problema y prever una posible solución o idea por donde
buscar la solución. Para ellos es necesario realizar el análisis de las fuerzas
que afectan al cuerpo en estudio. Éste tipo de análisis se denomina
\emph{Diagrama de cuerpo aislado}.

\subsubsection{Diagrama de cuerpo aislado}
\label{c.dca}

A continuación está el link para acceder al vídeo de la clase preparadas para el
tema desarrollado en la sección (\ref{c.dca}):

\begin{itemize}
  \item Diagrama de cuerpo aislado. Aplicación a un sistema en equilibrio \\
    \href{https://youtu.be/O173vt6zEno}{https://youtu.be/O173vt6zEno}
\end{itemize}


\subsection{2º Ley de Newton}
\label{c.2ln}

\begin{tcolorbox}[colback=black!5!white,colframe=red!35!black]
La 2º Ley de Newton establece que la resultante de las fuerzas que actúan sobre
un cuerpo es igual al producto de la masa del cuerpo por la aceleración que éste
adquiere: 
$$
\sum\vec{F} = m\cdot \vec{a}
$$
\end{tcolorbox}

% hablar sobre el significado de cada lado de la ecuación, dinámica/externo a
% cinemática/interno y la masa (interna)


\subsection{3º Ley de Newton o Ley de Acción y Reacción}
\label{c.3ln}

\begin{tcolorbox}[colback=black!5!white,colframe=red!35!black]
La 3º Ley de Newton establece que ante la acción ($\vec{A}$) que un cuerpo
realice a otro, éste le realiza una reacción ($\vec{Re}$) que es igual en
magnitud y dirección, pero de sentido contrario:
$$
\vec{A} = -\vec{Re}
$$
\end{tcolorbox}

%\subsubsection{Fuerza de roce o de fricción}

% INCLUIR EL GRÁFICO mu vs F Y EXPLICAR
% hablar sobre la balanza

% -------------------------------------------

\section{Aplicaciones de las Leyes de Newton}

% ---
\subsection{Balanza -- báscula}
\label{c.bb}

A continuación está el link para acceder al vídeo de la clase preparadas para el
tema desarrollado en la sección (\ref{c.bb}):

\begin{itemize}
  \item Aplicación de las Leyes de Newton: balanza -- báscula \\
    \href{https://youtu.be/vFnwwjrbMFc}{https://youtu.be/vFnwwjrbMFc}
\end{itemize}

\subsubsection*{Actividades (balanza)}
\small
\begin{enumerate}
  \item Realizar el análisis físico, en base a las Leyes de Newton, de una
  persona parada sobre una superficie plana cuando desea saltar. ¿Cómo se
  modifica su peso y la fuerza normal? ¿Es necesaria una fuerza extra para que
  la persona logre saltar? En caso afirmativo, ¿quién realiza esta fuerza?
  \item ¿Qué sucede si la superficie está inclinada?
  \item Realizar el desarrollo analítico para el caso de una persona ubicada
  sobre una balanza dentro del ascensor que desciende acelerado.
  \item Realizar el análisis de las siguiente situaciones y expresar en
  porcentaje el cambio ficticio de la masa ($m=55\kg$) que sufriría la persona
  por encontrarse en:\footnote{Se considera que el sistema de referencia es
  positivo en la dirección vertical ascendente.}
    \begin{enumerate}
      \item $\vec{a}=4.9\ms^2$.
      \item $\vec{a}=9.8\ms^2$.
      \item $\vec{a}=19.6\ms^2$.
      \item $\vec{a}=-4.9\ms^2$.
      \item Se corta el cable del ascensor.
      \item ¿Qué vector aceleración se necesita para que la balanza indique un
      10\% de aumento del peso?
    \end{enumerate}
    ¿Qué conclusión puede obtener sobre los porcentajes de variación de ``la
    masa'' respecto de las aceleraciones que experimenta la persona dentro del
    ascensor?
\end{enumerate}
\normalsize



% ---
\subsection{Plano inclinado}
\label{c.pi}

Dado el siguiente problema que se muestra en la Figura (\ref{f.prob-pi}): un cuerpo
sobre un plano inclinado ($\alpha=30\grm$), cuya masa es $m=10\kg$, es tirado hacia
arriba por una soga en la dirección del plano por una fuerza $\vec{F}=30\N$.
Entre el cuerpo y el plano existe fricción, siendo sus coeficientes:
$\mu_{e_{max}}=0.8$ para la fricción estática máxima y $\mu_{c}=0.4$ para la
cinética. Pregunta a responder: \emph{¿En qué condición se encuentra el cuerpo?}

\begin{figure}[!ht]
    \centering
  \setlength{\unitlength}{1mm}
  \begin{picture}(80,40)
  % --- plano ---
  \put(0,0){\line(3,1){80}}
  \put(0,0){\line(1,0){80}}
  \qbezier(18,0)(20,3)(18,6)
  \put(13,2){$\alpha$}
  % --- cuerpo ---
  \put(30,20){\line(3,1){15}}
  \put(32.9,11.0){\line(-1,3){3}}
  \put(48.0,16.0){\line(-1,3){3}}
  % --- fuerza que tira ---
  \linethickness{0.007mm}
  \put(46.3,21){\line(3,1){30}}
  \linethickness{2mm}
  \thicklines
  \put(46.3,21){\vector(3,1){15}}
  \put(55,27){$\vec{F}$}
  \end{picture}
  \caption{Esquema de la situación problemática sobre el plano inclinado.}
  \label{f.prob-pi}

\end{figure}

Al analizar el problema, existen 5 posibles condiciones en las que se puede
encontrar el cuerpo: subiendo por el plano (con equilibrio -MRU- o sin
equilibrio -MRUV-), quieto (estático) o bajando por el plano (con y sin
equilibrio), cada situación con su correspondiente coeficiente de fricción
(Tabla \ref{t.condicionpi}).

\begin{table}[!ht]
  \centering
\begin{tabular}{c|c|c|c}
  \hline\hline
  \multirow{2}{*}{Estado del cuerpo} & \multicolumn{2}{|c|}{Características del}
  & \multirow{2}{*}{Tipo de Fricción} \\ \cline{2-3}
  & Estado & Movimiento &\\
  \hline
  \multirow{2}{*}{Subiendo} & En equilibrio & MRU ($v\neq0$) &
  \multirow{2}{*}{cinética} \\
  & Sin equilibrio & MRUV \\
  \hline
  Quieto & En equilibrio & MRU ($v=0$) & estática \\
  \hline 
  \multirow{2}{*}{Bajando} & En equilibrio & MRU ($v\neq0$) &
  \multirow{2}{*}{cinética} \\
  & Sin equilibrio & MRUV \\
  \hline\hline
\end{tabular}
  \caption{Posibles condiciones en las que se encuentra el cuerpo. Además se
  identifican los coeficientes de fricción según la situación.}
  \label{t.condicionpi}
\end{table}

Para poder responder la pregunta, se siguen estos pasos:
\begin{enumerate}
  \item Identificar las fuerzas que actúan sobre el cuerpo, utilizar la 1º y 3º
    Leyes de Newton;
  \item Realizar el diagrama de cuerpos aislado de las fuerzas que interactúan
    con el cuerpo;
  \item Escoger el sistema de referencia a utilizar;
  \item Ubicar las fuerzas en el sistema de referencia sin considerar al cuerpo;
  \item Descomponer las fuerzas que no coincidan con los ejes coordenados del
    sistema escogido.
  \item Plantear la 2º Ley de Newton y analizar el problema.
\end{enumerate}

\subsubsection{Identificación de fuerzas según las leyes de Newton}

Las fuerzas que se identifican son la originada por la atracción gravitatorio de
la Tierra: el peso ($\vec{P}$), con dirección y sentido al centro de la Tierra.
La misma se aplica desde el centro de masa del cuerpo, normalmente es el centro
geométrico si el cuerpo es homogéneo.

La reacción producida por la acción que produce el peso sobre la superficie:
fuerza normal $N$, con dirección perpendicular a la superficie y sentido
saliente de la superficie. Aplicada desde la superficie de contacto (cuña) sobre
el cuerpo.

También la fuerza de fricción $f_r$, que es opuesta al movimiento o intento de
movimiento del cuerpo; para esta fuerza vamos a suponer que el cuerpo se
encuentra ascendiendo o intentando ascender sobre la cuña. La misma se aplica
sobre la superficie de contacto entre las dos superficies, la del cuerpo en
estudio y la de la cuña.

Notar que la fuerza normal y de fricción son consecuencia directa de la 3º Ley
de Newton.

\subsubsection{Diagrama de cuerpo aislado y sistema de referencia}

En la Figura (\ref{f.dcasr}) se representan las fuerzas identificadas
anteriormente sobre el cuerpo en estudio. Se especifican los puntos de
aplicación de las fuerzas. Notar que para la fuerza de roce o fricción, también
se especifica un punto, pero en realidad corresponde a todos los puntos de
contacto de ambas superficies. Las magnitudes de $\vec{P}$, $\vec{N}$ y
$\vec{F}$ están representadas a escala en la figura, para $\vec{f_r}$ es
aproximada ya que no se sabe que tipo fricción está actuando y además, si fuera
estática, la misma varía entre 0 y el máximo valor.

\begin{figure}[!ht]
    \centering
  \setlength{\unitlength}{1mm}
  \begin{picture}(80,40)
    % --- plano ---
    \put(0,0){\line(3,1){80}}
    \put(0,0){\line(1,0){80}}
    \qbezier(18,0)(20,3)(18,6)
    \put(13,2){$\alpha$}
    % --- cuerpo ---
    \put(30,20){\line(3,1){15}}
    \put(32.9,11.0){\line(-1,3){3}}
    \put(48.0,16.0){\line(-1,3){3}}
    % --- fuerza que tira ---
    \linethickness{0.007mm}
    \put(46.3,21){\circle*{1}}
    \put(46.3,21){\line(3,1){30}}
    \linethickness{2mm}
    \thicklines % todos los vectores en negrita
    \put(46.3,21){\vector(3,1){5}}
    \put(55,27){$\vec{F}$}
    % --- fuerzas de interacción ---
    \put(39.5,18){\circle*{1}}
    \put(39.5,18){\vector(0,-1){15}}
    \put(42,7){$\vec{P}$}
    \put(41.1,14){\circle*{1}}
    \put(41.1,14){\vector(-1,3){4}}
    \put(37,27){$\vec{N}$}
    \put(38,12.7){\circle*{1}}
    \put(38,12.7){\vector(-3,-1){7.5}}
    \put(31,06){$\vec{f_r}$}
    % --- Sistema de referencia ---
    \put(-10,10){\vector(3,1){35}}
    \put(25,23){$x$}
    \put(0,5){\vector(-1,3){12}}
    \put(-10,40){$y$}
    \put(-2.3,12.3){\circle*{1}}
    \put(-1,10){o}
  \end{picture}
  \caption{Diagrama de cuerpo aislado y sistema de referencia a utilizar. Se
  especifican las fuerzas que actúan sobre el cuerpo en estudio. Con un círculo
  se ubica al punto de aplicación de cada fuerza.}
  \label{f.dcasr}


\end{figure}

También está especificado el sistema de referencia a utilizar para describir el
problema. Se utiliza un sistema de referencia donde el sistema de
coordenadas ortogonales está rotado la misma inclinación del plano. De esta
forma los vectores $\vec{f_r}$, $\vec{N}$ y $\vec{F}$ coinciden con la dirección
de alguno de los ejes de coordenadas, simplificando su tratamiento posterior.

\subsubsection{Fuerzas y sistema de referencia}

A continuación se representa solamente las fuerzas ubicadas en el sistema de
coordenadas utilizado para describir el problema (Figura \ref{f.dfsr}). El
cuerpo en estudio queda representado como partícula puntual en el origen de
coordenadas.

\begin{figure}[!ht]
    \centering
  \setlength{\unitlength}{1mm}
  \begin{picture}(120,40)
    % --- Sistema de referencia rotado ---
    \put(17,39){(a)}
    \put(0,5){\vector(3,1){35}}
    \put(35,13){$x$}
    \put(21,-3){\vector(-1,3){12}}
    \put(11,33){$y$}
    \put(17,8.5){o}
    \put(16.5,10.5){\circle*{1}}
    % --- fuerzas ---
    \thicklines
    \put(16.5,10.5){\vector(3,1){5}}
    \put(23,7){$\vec{F}$}
    \put(16.5,10.5){\vector(-3,-1){7.5}}
    \put(3,1){$\vec{f_r}$}
    \put(16.5,10.5){\vector(0,-1){10}}
    \put(13,-3){$\vec{P}$}
    \put(16.5,10.5){\vector(-1,3){2.5}}
    \put(15,19){$\vec{N}$}
    %
    % --- Sistema de referencia plano ---
    \put(77,39){(b)}
    \thinlines
    \put(55,13){\vector(1,0){40}}
    \put(95,11){$x$}
    \put(73,0){\vector(0,1){35}}
    \put(70,35){$y$}
    \put(73.5,10.5){o}
    \put(73,13){\circle*{1}}
    % --- fuerzas ---
    \thicklines
    \put(73,13){\vector(1,0){5}}
    \put(78,9){$\vec{F}$}
    \put(73,13){\vector(0,1){8.6}}
    \put(75,20){$\vec{N}$}
    \put(73,13){\vector(-1,0){7}}
    \put(61,9){$\vec{f_r}$}
    \put(73,13){\vector(-1,-3){4}}
    \put(69,-3){$\vec{P}$}
  \end{picture}
  \caption{Diagrama de fuerzas sobre el sistema de coordenadas: (a) orientado
  según el plano inclinado; (b) sin rotación. El punto representa al cuerpo.}
  \label{f.dfsr}


\end{figure}

\subsubsection{Descomposición de fuerzas}

Según la elección del sistema de referencia, la única fuerza que no coincide con
los ejes de coordenadas es la fuerza del peso $\vec{P}$. Por lo tanto es
necesario descomponer ésta fuerza en sus componentes en la dirección en $x$
($\vec{P_x}$) y en $y$ ($\vec{P_y}$). Esto se realiza debido a que en este
momento solo sabemos sumar vectores que sean paralelos o perpendiculares, no
sabemos sumar vectores \emph{oblicuos}. Al descomponer la fuerza peso, es como
si la fuerza peso ya no exista más y en cambio se generan dos vectores nuevos
pero que son perpendiculares entre sí.


\begin{figure}[!ht]
    \centering
  \setlength{\unitlength}{1mm}
  \begin{picture}(120,40)
    % --- plano ---
    \put(37,39){(a)}
    \put(0,0){\line(3,1){60}}
    \put(3,3){A}
    \put(0,0){\line(1,0){60}}
    \put(60,-3){B}
    \qbezier(18,0)(20,3)(18,6)
    \put(13,2){$\alpha$}
    % --- direcciones ---
    \put(15,12){\line(1,0){45}}
    \put(60,9){B'}
    \put(36,-3){\line(0,1){35}}
    \put(37,30){C}
    \put(39.9,0.3){\line(-1,3){10}}
    \put(26,30){D}
    % --- ángulos ---
    \qbezier(50,12)(52,14.5)(50,16.7)
    \put(46,13){$\alpha$}
    \qbezier(37.8,7)(42,9)(42,12)
    \put(37.8,8.7){$\beta$}
    \qbezier(36,2)(37,1)(39,2)
    \put(36,3){$\gamma$}
    %
    % --- ubicación en sist. coord. ---
    \put(93,39){(b)}
    \thinlines
    \put(75,23){\vector(1,0){40}}
    \put(115,21){$x$}
    \put(93,0){\vector(0,1){35}}
    \put(90,35){$y$}
    \put(93.5,20.5){o}
    \put(93,23){\circle*{1}}
    \qbezier(87.5,6)(90,4)(93,6)
    \put(89,8){$\alpha$}
    % --- fuerzas ---
    \thicklines
    \put(93,23){\vector(-1,-3){7}}
    \put(85,-3){$\vec{P}$}
    \put(93,23){\vector(0,-1){21.2}}
    \put(95,9){$\vec{P}_y$}
    \put(93,23){\vector(-1,0){7}}
    \put(85,25){$\vec{P}_x$}
    \linethickness{0.07mm}
    \put(93,1.8){\line(-1,0){7}}
    \put(86,23){\line(0,-1){21.2}}
    % --- Sistema de referencia oblicuo ---
    \thinlines
    \put(-10,10){\vector(3,1){20}}
    \put(10,15){$x$}
    \put(0,5){\vector(-1,3){7}}
    \put(-6,25){$y$}
  \end{picture}
  \caption{(a) Diagrama geométrico para identificar el ángulo en el sistema de
  fuerzas. (b) El vector peso, sus componentes y el ángulo utilizado.}
  \label{f.dgeom}


\end{figure}

Para la deducción del ángulo utilizado para obtener las componentes del peso
($P_x$ y $P_y$), se utiliza la Figura (\ref{f.dgeom}-a). Se presentan 5 rectas:
A, B, B', C y D. Las rectas B y B' son paralelas; C es perpendicular a B' y B; D
es perpendicular a A. El ángulo indicado como $\alpha$ y formado por las rectas
A y B es el mismo ángulo que el formado por A y B'. Como A y D son
perpendiculares, se deduce que $\beta$ es el complemento de $\alpha$:
\[
  \beta= 90\grm - \alpha.
\]
Luego, como B' y C son perpendiculares, $\gamma$ resulta el complemento de
$\beta$: 
\[
  \gamma = 90\grm - \beta. 
\]
Por lo tanto, reemplazando se obtiene:
\begin{align} \label{e.1}
  \gamma &= 90\grm - \beta \nonumber \\
  &= 90\grm - (90\grm - \alpha) \nonumber \\
  \Aboxed{\gamma &= \alpha}
\end{align}

El ángulo $\gamma$ se forma entre las rectas D y C, correspondientes a la
dirección de la normal o eje $y$ y la dirección del peso, respectivamente. Por
lo tanto, el ángulo $\gamma$, que es igual al ángulo de inclinación del plano
$\alpha$ (Ecuación \ref{e.1}), esta ubicado entre la dirección del peso y el eje
$y$ como se observa en la Figura (\ref{f.dgeom}-b).

Para calcular las componentes del peso, se utiliza el triángulo rectángulo que
contiene al ángulo $\alpha$. Utilizando las relaciones trigonométricas
$\sen(\alpha) = \dfrac{CO}{H}$ y $\cos(\alpha)=\dfrac{CA}{H}$, e identificando a
$H\equiv |\vec{P}|$, $CO\equiv |\vec{P}_x|$ y $CA \equiv |\vec{P}_y|$, podemos
despejar y obtener las componentes:
\begin{align} \label{e.2}
  \sen(\alpha) = \dfrac{P_x}{P} \implies \Aboxed{P_x &= P \sen(\alpha)}
  \nonumber \\ \\
  \cos(\alpha) = \dfrac{P_y}{P} \implies \Aboxed{P_y &= P \cos(\alpha)} \nonumber
\end{align}

Para nuestro caso en particular, como $m=10\kg$ ($P=m\times g=98\N$) y
$\alpha=30\grm$, utilizando las Ecuaciones (\ref{e.2}) obtenemos:
\begin{align} \label{e.ps}
  P_x & = 98\N\times \sen(30\grm) = 49\N \\
  P_y & = 98\N\times \cos(30\grm) =84.9\N \nonumber
\end{align}

\subsubsection{Planteo de la 2º Ley de Newton y análisis del problema}

Debido a la imposibilidad de sumar vectores que no sean paralelos o
perpendiculares, vamos a utilizar las componentes del peso y resolver a través
del planteo de la 2º Ley de Newton en componentes $x$ e $y$.

\paragraph{Dirección $y$ -- perpendicular al plano inclinado\\}

El cuerpo por sí solo no va a despegarse ni entrar en la superficie de la cuña,
por lo tanto es válido plantear que en esta dirección hay equilibrio ($a_y=0$):
\begin{equation}
  N - P_y = m a_y = 0
\end{equation}
Al despejar obtenemos el valor de la fuerza normal:
\begin{equation} \label{e.n}
  N = P_y = 84.9\N
\end{equation}

\paragraph{Dirección $x$ -- paralelo al plano inclinado\\}

En esta dirección es donde se debe realizar el análisis del estado de movimiento
del cuerpo, por lo tanto se va a generalizar el planteo sin suponer equilibrio:
\begin{equation*}
  F - f_r - P_x = m a_x,
\end{equation*}
o escrita en términos de las fuerzas que contribuya a que suba $F$ y las que no
$f_r$ y $P_x$:
\begin{equation} \label{e.2lnx}
  F - (f_r + P_x) = m a_x.
\end{equation}

Para realizar el análisis nos vamos a basar en la Tabla (\ref{t.condicionpi}).

Como no sabemos cuál es el régimen de fricción que se va a encontrar el cuerpo,
calculamos ambos casos. Para calcular la fuerza de fricción, recordamos que
\begin{equation}
  f_{re_{max}} = \mu_{e_{max}} N \quad \mathrm{y} \quad f_{rc}=\mu_{c} N,
\end{equation}
y que la misma se orienta según la dirección opuesta al movimiento o intento de
movimiento en la superficie de contacto del cuerpo en estudio (Figura
\ref{f.dcasr}). Conociendo el valor de $N$ (Ecuación \ref{e.n}) y los valores de los
coeficientes dados al comienzo, calculamos las fuerzas de roce:
\begin{align} \label{e.fr}
  f_{re_{max}} &= 67.9\N \\
  f_{rc} &= 33.9\N \nonumber
\end{align}

A continuación vamos a analizar cada una de las posibles situaciones que se
pueda encontrar el cuerpo.

\paragraph{Sin equilibrio y ascendiendo.}
Suponemos que el cuerpo se encuentra fuera de equilibrio y ascendiendo sobre la
cuña. En esta condición, la fuerza de fricción corresponde al roce cinético y la
aceleración debería ser positiva.
Entonces, utilizando los datos de las Ecuaciones (\ref{e.fr}), (\ref{e.ps}) y la
Ecuación (\ref{e.2lnx}), obtenemos:
\begin{align*}
  30\N - (33.9\N + 49\N) &= m a_x \\
  30\N - (82.9\N) & = 10\kg\; a_x \\
  -5.29\ms^2 & =  a_x < 0 
\end{align*}
Este resultado está en contradicción con nuestra suposición ya que la
aceleración debía ser positiva. {\bf Por lo tanto el cuerpo no se puede
encontrar en éste estado.}

\paragraph{En equilibrio y ascendiendo.}
Suponemos que el cuerpo se encuentra en equilibrio y ascendiendo sobre la rampa.
En este estado corresponde utilizar la fuerza de fricción cinética y la
aceleración debería ser nula. Entonces utilizando las mismas ecuaciones que
antes se obtiene:
\begin{align*}
  30\N - (33.9\N + 49\N) &= 0 \\
  30\N - (82.9\N) & = 0 \\
  -52.9\N & \neq 0  
\end{align*}
Lo cual es incoherente con la suposición. {\bf Por lo tanto el cuerpo no puede
encontrarse en equilibrio y ascendiendo.}


\paragraph{Estático.}


MEJORAR LA DESCRIPCIÓN E INCLUIR TODAS LAS POSIBILIDADES: ASCIEDE, DESCIENDE O
ESTÁ QUIETO!!!

Suponemos que el cuerpo se encuentra estático, por lo tanto en equilibrio. La
aceleración debe ser nula y se utiliza la fricción estática $f_{re}$. Tener
presente que la fuerza de roce estática puede tomar cualquier valor entre $0\N$
y $f_{re_{max}}$ calculado anteriormente. Nuevamente, con las mismas ecuaciones
y resolviendo para $f_{re}$ se obtiene:
\begin{align*}
  30\N - (f_{re} + 49\N) &= 0 \\
  30\N - 49\N &= f_{re}  \\
  -19\N &=  f_{re}
\end{align*}
Este resultado no esta mal, sino que nuestra suposición inicial que el cuerpo se
encuentra en movimiento ascendente o intentando ascender estaba equivocada. Esto
significa que el cuerpo sí se puede encontrar en equilibrio estático y que la
fuerza de fricción estática debe tomar éste valor encontrado y ser dirigida
hacia arriba de la cuña. Es decir que el cuerpo está intentando \emph{caer} por
el plano inclinado. {\bf Por lo tanto el cuerpo se puede encontrar estático si
la fuerza de fricción estática vale $\mathbf{f_{re} = -19\N}$ dirigida en
sentido ascendente sobre la cuña.}

\paragraph{Sin equilibrio y descendiendo.}
Suponemos que se encuentra fuera de equilibrio y descendiendo, la fuerza de
roce es cinética y debe ser negativa (por nuestra elección original, ya que el
cuerpo se mueve ahora hacia abajo en la cuña), y la aceleración debe ser
negativa.  Utilizando las mismas ecuaciones y teniendo en cuenta lo anterior
resulta:
\begin{align*}
  30\N - (-33.9\N + 49\N) &= m a_x \\
  30\N - (15.1\N) & = 10\kg\; a_x \\
  1.49\ms^2 & =  a_x > 0
\end{align*}
Nuevamente se contradice la suposición, ya que la aceleración debía ser
negativa. {\bf Por lo tanto el cuerpo no se puede encontrar fuera de equilibrio
y descendiendo por la rampa.}

\paragraph{En equilibrio y descendiendo.}
Suponemos que el cuerpo está en equilibrio pero desciende por el plano inclinado
a velocidad constante. Corresponde utilizar la fricción cinética y la
aceleración debe ser nula. Utilizando las ecuaciones con las condiciones
presentes obtenemos:
\begin{align*}
  30\N - (-33.9\N + 49\N) &= 0 \\
  30\N - (15.1\N) & = 0 \\
  14.9\N & \neq 0  
\end{align*}
Lo cual contradice la suposición que el cuerpo se encuentra en MRU. {\bf Por lo
tanto el cuerpo no se puede encontrar en equilibrio y descendiendo por la
cuña.}

\subsubsection{Conclusiones finales}

Según los análisis realizados, sólo un estado es posible para el cuerpo. Tener
presente que en los casos que se supone que el cuerpo se encuentra fuera de
equilibrio, por ejemplo si el cuerpo asciende, puede darse el caso que el cuerpo
se encuentre con velocidad ascendente por la rampa pero con aceleración negativa.
Esto quiere decir que el cuerpo va a terminar descendiendo luego de cierto
tiempo. Contradictorio a lo supuesto que no debería dejar de ascender. Por tal
motivo no es posible para obtener resultados favorables en caso de querer subir
al cuerpo por el plano inclinado, además de depender de la velocidad original
que tenga el cuerpo. Se puede realizar el mismo análisis para el caso que se
encuentre descendiendo y descartar por el mismo motivo. 

\cuadro{Por lo tanto, respondiendo a la pregunta original: \emph{¿En qué
condición se encuentra el cuerpo?}, podemos afirmar que el cuerpo se encuentra
en equilibrio estático, intentando descender sobre el plano inclinado. En
referencia al régimen de fricción, se encuentra en el estado de roce estático
sin haber alcanzado su máximo valor posible.
}

En la siguiente Tabla (\ref{t.resultado-pi}) se resumen todos los resultado
alcanzados durante el análisis:

\begin{table}[!h]
  \centering
\begin{tabular}{c|c|c|c|c}
  \hline\hline
  \multirow{2}{*}{Estado del cuerpo} & \multicolumn{2}{|c|}{Características del}
  & \multirow{2}{*}{Tipo de Fricción} & \multirow{2}{*}{Posibilidad} \\
  \cline{2-3}
  & Estado & Movimiento &\\
  \hline
  \multirow{2}{*}{Subiendo} & En equilibrio & MRU ($v\neq0$) &
  \multirow{2}{*}{cinética} & $\nexists$ \\
  & Sin equilibrio & MRUV && $\nexists$ \\
  \hline
  \multirow{2}{*}{Quieto} & \multirow{2}{*}{En equilibrio} & \multirow{2}{*}{MRU
  ($v=0$)} & \multirow{2}{*}{estática} & $\exists$ con $f_{re}<0$ \\
  &&&& (ascendente)\\
  \hline 
  \multirow{2}{*}{Bajando} & En equilibrio & MRU ($v\neq0$) &
  \multirow{2}{*}{cinética} & $\nexists$  \\
  & Sin equilibrio & MRUV && $\nexists$  \\
  \hline\hline
\end{tabular}
  \caption{Resultados de los análisis en que se encuentra el cuerpo.}
  \label{t.resultado-pi}
\end{table}


\subsubsection*{Actividades (plano inclinado)}
\small
Realizar el análisis cuando la fuerza con la cual el cuerpo es tirado
ascendentemente sobre el plano inclinado toma los siguiente valores: 
\begin{enumerate}
  \item $\vec{F}=40\N$.
  \item $\vec{F}=49\N$.
  \item $\vec{F}=60\N$.
  \item Considerar $\alpha=45\grm$.
    \begin{enumerate}
      \item Encontrar el valor de la fuerza $\vec{F}$ tal que el cuerpo se
	encuentra en equilibrio y la fuerza de roce estática es nula
	($f_{re}=0\N$).
      \item ¿Es posible que el cuerpo se encuentre en equilibrio si
	$\vec{F}=0\N$? Analizar y justificar.
    \end{enumerate}
  \item Analizar la situación cuando la inclinación del plano es
  $\alpha=90\grm$.
\end{enumerate}
\normalsize


